\chapter{Itemized Lists and Quotes}

Itemized lists work as one would expect. They are properly line-spaced. Refer
to the tutorial for more details. In general, it is better to start a new 
paragraph (i.e. place a blank line in the source file) at the start of a new 
enumerate block. This ensures that the line-spacing options are activated 
correctly.

Line spacing of items in the list and the spacing of the list with respect to
the surrounding textual matter is set as per the institute guidelines.

\section{Unnumbered lists}
These are the standard \textquote{bulleted} lists that one is familiar with.
\begin{enumerate}[blt]
    \item this
    \item is
    \item a bulletted
    \item list
\end{enumerate}

this is the next paragraph

\begin{enumerate}[dmd]
    \item this
    \item is
    \item a dashed
    \item list
\end{enumerate}

and this is the next paragraph

\begin{enumerate}[dsh]
    \item this
    \item[$\blacksquare$] is
    \item[\textbullet] a mixed
    \item list
\end{enumerate}

and this is the next paragraph

\begin{enumerate}[str]
    \item this
    \item is
    \item[\textbullet] a mixed
    \item list
\end{enumerate}

and this is the next paragraph

\begin{enumerate}[sqr]
    \item this
    \item is
    \item square
    \item list
\end{enumerate}

and this is the next paragraph 

\begin{enumerate}[bsq]
    \item this
    \item is
    \item square
    \item list
\end{enumerate}

\section{Numbered lists}
These are the standard numbered lists that one might be familiar with.
\begin{enumerate}
    \item this
    \item is a
    \item numbered
    \item list
\end{enumerate}

this is the next paragraph

\begin{enumerate}[i.]
    \item this
    \item is a
    \item roman-numbered
    \item list
\end{enumerate}


this is the next paragraph

\begin{enumerate}[I{]}]
    \item this
    \item is a
    \item capitalized roman-numbered
    \item list
\end{enumerate}

this is the next paragraph

\begin{enumerate}[a)]
    \item this
    \item is an
    \item alphabetical
    \item list
\end{enumerate}

this is the next paragraph

\begin{enumerate}[A\}]
    \item this
    \item is the
    \item last
    \item This is what that author said and it is rather important. That is the
    reason I am quoting the entire passage here for reference and completeness
    \begin{enumerate}
        \item This is what that author said and it is rather important. That is
the reason I am quoting the entire passage here for reference and completeness
        \item  This is what that author said and it is rather important. That is
the reason I am quoting the entire passage here for reference and completeness
    \end{enumerate}
\end{enumerate}

Note that one can nest lists as well as shown above

\section{More complex lists made easy}
\begin{enumerate}[{Step rsc{]}}, leftmargin=2cm]
    \item reflect over the research done
    \item review ones notes, writings, lab journal and research articles
    \item stop overthinking and/or procrastinating
    \item start writing
    \item realize that there is much to do
    \begin{enumerate}[{i\}}]
        \item fear not
        \item[$\star$] persevere
        \item continue writing
    \end{enumerate}
    \item success!
\end{enumerate}


\section{Quotes}

Very useful for quoting text from other works. See the tutorial for more
details and references.

An inline-quote, one that is along with the main material is placed like this
\textquote{this is a quote}. Sometimes we want to mention the author as well
\textquote[Author of the quote]{this is what that author wrote}

Sometimes, it is better to quote larger passages of text in the display mode
like this

\begin{displayquote}
This is what that author said and it is rather important. That is the reason I
am quoting the entire passage here for reference and completeness
\end{displayquote}

this is the next paragraph. By specifiying the optional argumnet, the author
name can be set as well:

\begin{displayquote}[Authorname]
This is what that author said and it is rather important. That is the reason I
am quoting the entire passage here for reference and completeness
\end{displayquote}

\section{Footnotes}
One might find the need to include some relevant contextual information in the
material but find it unneccessarily intrusive to do so in the running text
matter itself. In such cases, or whenever else the author feels the need to do
so, a footnote may be deployed thus \footnote{see, this is the footnote I was
talking about. It is spaced appropriately. Best to use these sparingly since
they detract from the general flow of the document. Some thoughtful
restructuring of the material can help reduce the need for footnotes}