\chapter{Few cool things}
\section{Order of operations}
When more than one operator appears in an expression, the order of evaluation depends on the rules of precedence. Python follows the same precedence rules for its mathematical operators that mathematics does. The acronym PEMDAS is a useful way to remember the order of operations: Parentheses have the highest precedence and can be used to force an expression to evaluate in the order you want. Since expressions in parentheses are evaluated first, $2* (3-1)$ is $4$, and $(1+1)**(5-2)$ is $8$. You can also use parentheses to make an expression easier to read, as in $(\text{minute} * 100)/60$, even though it does not change the result. Exponentiation has the next highest precedence, so $2**1+1$ is $3$ and not $4$, and $3*1**3$ is $3$ and not $27$. Multiplication and Division have the same precedence, which is higher than Addition and Subtraction, which also have the same precedence. So $2*3-1$ yields $5$ rather than $4$, and $2/3-1$ is $-1$, not 1 (remember that in integer division, $2/3=0$). Operators with the same precedence are evaluated from left to right. So in the expression $\text{minute}*100/60$, the multiplication happens first, yielding $5900/60$, which in turn yields $98$. If the operations had been evaluated from right to left, the result would have been $59*1$, which is $59$, which is wrong.

\section{Comments}
Comments As programs get bigger and more complicated, they get more difficult to read. Formal languages are dense, and it is often difficult to look at a piece of code and figure out what it is doing, or why. For this reason, it is a good idea to add notes to your programs to explain in natural language what the program is doing. These notes are called comments, and they are marked with the \# symbol, you have seen throughout the book green color comments has been made to inform users on what each line doing.