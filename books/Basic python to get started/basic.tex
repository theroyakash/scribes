\chapter{Basic Python Introduction}
\section{What is a program?}
\noindent A program is a sequence of instructions that specifies a computation. It may be something mathematical problem maybe like finding some roots of a polynomial but it also can be a symbolic representation like reading and loading a text into memory from a text file. \\

\noindent For example following is a \texttt{text} file, to load it into the memory you might write code like this

\noindent\begin{minipage}{\linewidth}
\lstinputlisting[style=python]{code_examples/read.py}
\end{minipage}

\noindent The details looks different in different languages but a few basic instruction appears in every languages.\\

\noindent For about every language you might do these following operations to complete a specific task in hand meaning completing a computation. Your program may structure like this
\begin{itemize}
    \item {\sl Input}: Get data from the keyboard, a file, or some other device.
    \item {\sl Output}: Display data on the screen or send data to a file or other device.
    \item {\sl Math}: Perform basic mathematical operations like addition and multiplication.
    \item {\sl Conditional execution}: Check for certain conditions and execute the appropriate sequence of statements.
    \item {\sl Repetition of small steps}: Perform some action repeatedly, usually with some variation.
\end{itemize}

Believe it or not, that’s pretty much all there is to make a program. Every program you’ve ever used, no matter how complicated that is, is made up of instructions that look more or less steps like these. 
We can describe programming as the process of breaking a large, complex task into smaller and smaller subtasks until the subtasks are simple enough to be performed with one of these basic instructions. That may be a little vague, but you will know about this later you study algorithms.

\subsection{Formal and natural languages}
\noindent Natural languages are the languages that people speak, such as English, Spanish, and French. They were not designed by people (although people try to impose some order on them); they evolved naturally.\\

But formal languages are languages that are designed by people for specific applications. For example, the notation that mathematicians use is a formal language that is particularly good at denoting relationships among numbers and symbols. Chemical Scientists use a formal language to represent the chemical structure of molecules. And most importantly:
Programming languages are formal languages that have been designed to express computations to be performed on a specialized computing device.
Formal languages tend to have strict rules about syntax. For example, 3 + 3 = 6 is a syntactically correct mathematical statement, but 3=+6\$ is not. H2O is a syntactically correct chemical name, but 2Zz is not.

Syntax rules come in two flavors, pertaining to tokens and structure. Tokens are the basic elements of the language, such as words, numbers, and chemical elements. One of the problems with 3=+6$ is that $ is not a legal token in mathematics (at least as far as we know). Similarly, 2Zz is not legal because there is no element with the abbreviation Zz.
The second type of syntax error pertains to the structure of a statement—that is, the way the tokens are arranged. The statement 3=+6\$ is structurally illegal because you can’t place a plus sign immediately after an equal sign. Similarly, molecular formulas have to have subscripts after the element name, not before.

\subsection{First program}
Traditionally, the first program written in a new language is called “Hello, World!” because all it does is display the words, “Hello, World!” In Python, it looks like this:

\noindent\begin{minipage}{\linewidth}
\lstinputlisting[style=python]{code_examples/hello.py}
\end{minipage}

\noindent This is an example of a print statement, which doesn’t actually print anything on paper. It displays a value on the screen. In this case, the result is the words \texttt{Hello world hehe}.\\

\noindent The quotation marks in the program mark the beginning and end of the value; they don’t appear in the result.
Some people judge the quality of a programming language by the simplicity of the \texttt{“Hello, World!”} program. By this standard, Python does about as well as possible.

