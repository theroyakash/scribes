\documentclass[12pt, a4paper, onecolumn]{exam}
\usepackage{amsmath}
\usepackage{amssymb}
\usepackage{graphicx}
\usepackage{pdfpages}
\usepackage[linesnumbered,ruled]{algorithm2e}
\graphicspath{ {./images/} }
\usepackage[lmargin=71pt, tmargin=1.2in]{geometry}  %For centering solution box
% \lhead{Leaft Header\\}
% \rhead{Right Header\\}
% \chead{\hline} % Un-comment to draw line below header
\thispagestyle{empty}   %For removing header/footer from page 1

\usepackage{tcolorbox}

\tcbuselibrary{theorems}

\newtcbtheorem[number within=section]{theo}{Theorem}%
{colback=green!5,colframe=green!35!black,fonttitle=\bfseries}{th}

\newtcbtheorem[number within=section]{hypothesis}{Hypothesis}%
{colback=red!5,colframe=red!35!black,fonttitle=\bfseries}{th}

\newtcbtheorem[number within=section]{define}{Definition}%
{colback=blue!5,colframe=blue!35!black,fonttitle=\bfseries}{th}

\newcommand{\RP}{\ensuremath{\mathsf{RP}}}
\newcommand{\expect}[1]{\ensuremath{\mathbb{E}[#1]}}
\newcommand{\dx}{\mathrm{d}x}
\newcommand{\NP}{\ensuremath{\mathsf{NP}}}
\newcommand{\co}{\ensuremath{\mathsf{co}\mbox{-}}}
\newcommand{\NL}{\ensuremath{\mathsf{NL}}}

\begin{document}

\begingroup  
    \centering
    \LARGE Security in Cyber Physical System\\
    \LARGE Tutorial \# 1\\[0.5em]
    \large \today\\[0.5em]
    \large \textsc{Roy, Akash} $\mid$ CS22M007\par
    \large \textsc{M.Tech CS} $\mid$ \textsc{Indian Institute of Technology, Madras}\par
\endgroup
\pointsdroppedatright %Self-explanatory
\printanswers
\renewcommand{\solutiontitle}{\noindent\textbf{Ans:}\enspace}   %Replace "Ans:" with starting keyword in solution box

\begin{questions}

    \question What are the mathematical properties that are used for the secure exchange of keys (hint: in terms of symmetry equivalence etc.)?

    \begin{solution}
        For symmetry equivalence during the key exchange we have the followings a generator number $g$, a large prime number $n$, the two exchanging parties have two private keys $a, b$.

        The goal is to establish a secret key that's same for both the parties (symmetric key exchange).

        Suppose Alice is a party that has private key $a$, Bob is another party who has private key $b$. Alice will compute $g^a \text{ mod } n$. Bob will compute $g^b \text{ mod } n$. These two can be exchanged via public non-encrypted channel because finding what is $a$ from $g^a \text{ mod } n$ is computationally extremely hard.

        Now Alice has $g^b \text{ mod } n$, multiplying with $g^a \text{ mod } n$, Alice will get $g^{ab} \text{ mod } n$. Similarly Bob will multiply $g^a \text{ mod } n$ with $g^b \text{ mod } n$, then Bob will get $g^{ab} \text{ mod } n$.

        Now this number $g^{ab} \text{ mod } n$ is a secret value that is not in the public domain and can not be derived from all the variables available in the public domain. Thus giving us a symmetric key $g^{ab} \text{ mod } n$.

        So the properties are following
        \begin{itemize}
            \item Given a function $f$, and two keys $a, b$, and some operation
            $\text{op}$, $f(a) \text{ op } f(b) = f(b) \text{ op } f(a)$
            \item $f^{-1}$ is not easy to calculate.
        \end{itemize}
        
    \end{solution}

    \question Discuss how the above function can be used as your algorithm for the secure exchange of keys.

    \begin{solution}
        Suppose I've one function $f$ and have the following properties
        
        \begin{itemize}
            \item Given a function $f$, and two keys $a, b$, and some operation
            $\text{op}$, $f(a) \text{ op } f(b) = f(b) \text{ op } f(a)$
            \item $f^{-1}$ is not easy to calculate.
        \end{itemize}

        Now our key-exchange algorithm would be the following

        \begin{algorithm}[H]
            \SetKwInOut{Input}{Input}
            
            \Input{Private variable $a$ for Alice and variable $b$ for Bob, and one suitable operation \textsc{OP}}
            \BlankLine
            Calculate $f(a)$ from Alice's end.\\
            Calculate $f(b)$ from Bob's end.\\
            Exchange $f(a)$ and $f(b)$ with each other\\
            Use $f(a) \textsc{ OP } f(b)$ as the symmetric key for encryption.
            \caption{\textsc{Key-Exchange Algorithm}}
        \end{algorithm}
        
    \end{solution}
\end{questions}
\end{document}