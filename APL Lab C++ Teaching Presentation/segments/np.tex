
\begin{frame}
    \frametitle{Disjoint Support}

    Let's define pair $\langle x, y \rangle$ as ``$y$ was drawn according to
    $D_{n,x,\phi}$''. For a parameterized distributional problem $(\mathcal{L},
        \mathcal{D})$ we define

    \begin{align*}
        L_{\text{ds}} = \{\langle x, y \rangle \: \vert \: y \in \mathcal{L} \text{ and } \vert \: y \: \vert \leq \text{poly}(\mid x\mid)\}
    \end{align*}

\end{frame}

\begin{frame}
    \frametitle{Disjoint Support Continued, Reducibility}

    With this notion of $L_{\text{ds}}$ we define the notion of reducibility.

    \begin{define}{Notion of Reduction}{defn3}
        \textit{Let $\mathcal{L}$, $\mathcal{D}$ and $\mathcal{L}$', $\mathcal{D}$' be
            two parameterized distributional problem. We say $\mathcal{L}$, $\mathcal{D}$
            polynomial time smoothed reduces to $\mathcal{L}$', $\mathcal{D}$' if there exist
            a polynomial time computable function $f$ such that following holds}

    \end{define}

    \begin{itemize}
        \item $\langle x, y \rangle \in L_{\text{ds}}$ if and only if $f(\langle x, y \rangle; n, \phi) \in L_{\text{ds}}'$
    \end{itemize}

\end{frame}

\begin{frame}
    \frametitle{Disjoint Support Continued, Reducibility}

    \begin{define}{Notion of Reduction}{defn3}
        \textit{Let $\mathcal{L}$, $\mathcal{D}$ and $\mathcal{L}$', $\mathcal{D}$' be
            two parameterized distributional problem. We say $\mathcal{L}$, $\mathcal{D}$
            polynomial time smoothed reduces to $\mathcal{L}$', $\mathcal{D}$' if there exist
            a polynomial time computable function $f$ such that following holds}

    \end{define}

    \begin{itemize}
        \item $\langle x, y \rangle \in L_{\text{ds}}$ if and only if $f(\langle x, y \rangle; n, \phi) \in L_{\text{ds}}'$

        \item There exist a polynomial $p$ and $m$ such that for every $n, x, \phi, y' \in$
              supp $D_{m(n), f_1(\langle x, y \rangle; n, \phi), \phi}$ we have
              \begin{align*}
                  \displaystyle\sum_{y:f_1(\langle x, y \rangle; n, \phi), \phi = y'} D_{n,x,\phi} (y) \leq p(n) * D_{m(n), f_1(\langle x, y \rangle; n, \phi), \phi} (y')'
              \end{align*}
    \end{itemize}

\end{frame}

\begin{frame}
    \frametitle{$\textsf{PComp}_{\textsf{para}}$ and $\textsf{Dist-NP}_{\textsf{para}}$}

    We'll define the smoothed analog of classical worst case \textsf{NP} and
    average-case \textsf{Dist-NP} called $\textsf{Dist-NP}_{\textsf{para}}$. But first define $\textsf{PComp}_{\textsf{para}}$.

    \begin{define}{$\textsf{PComp}_{\textsf{para}}$}{defn 4}
        \textit{A parameterized family of distribution $\in$ $\textsf{PComp}_{\textsf{para}}$, if the cumulative probability}
        
        \begin{align*}
            F_{D_{n, x, \phi}} = \displaystyle\sum_{z \leq x} D_{n, x, \phi}
        \end{align*}

        \textit{can be computed in polynomial time given $n, x, \phi$ in binary.}
    \end{define}

\end{frame}

\begin{frame}
    \frametitle{$\textsf{PComp}_{\textsf{para}}$ and $\textsf{Dist-NP}_{\textsf{para}}$}

    
    \begin{center}
        $z \leq x$ means $z$ comes lexicographically before \(x\) or \(z = x\).
        
        With definiton 3.4 we define $\textsf{Dist-NP}_{\textsf{para}}$
    \end{center}

\end{frame}

\begin{frame}
    \frametitle{$\textsf{Dist-NP}_{\textsf{para}}$}

    
    \begin{define}{$\textsf{Dist-NP}_{\textsf{para}}$}{defn5}
        $\textsf{Dist-NP}_{\textsf{para}}$ = \{($L$, $\mathcal{D}$) $\mid$ $L \in \textsf{NP}$ and $\mathcal{D} \in \textsf{PComp}_{\textsf{para}}$\}
    \end{define}

\end{frame}

\begin{frame}
    \frametitle{$\textsf{Dist-NP}_{\textsf{para}}$ has Complete Problems}

    
    \begin{theo}{$\textsf{Dist-NP}_{\textsf{para}}$ completeness}{theo22}
        $\textsf{Dist-NP}_{\textsf{para}}$ has Complete Problems. Bounded Halting is one such example.
    \end{theo}

\end{frame}

\begin{frame}
    \frametitle{Proof}

    First we define Bounded Halting Problem.

    \begin{define}{Bounded Halting}{defnbnhalt}
        \textsf{BH} = \{ $\langle g, x, 1^t \rangle$ \(\mid\) the non deterministic Turing machine with Gödel number $g$ accepts x within t steps\}
    \end{define}

\end{frame}


\begin{frame}
    \frametitle{Proof}

    First we define Bounded Halting Problem.

    \begin{define}{Bounded Halting}{defnbnhalt}
        \textsf{BH} = \{ $\langle g, x, 1^t \rangle$ \(\mid\) the non deterministic Turing machine with Gödel number $g$ accepts x within t steps\}
    \end{define}

    \textbf{Note}: Think of Gödel Number $g$ is an encoding of the Turing Machine.

\end{frame}

\begin{frame}
    \frametitle{Lemma}Support

    

\end{frame}