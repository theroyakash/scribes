\documentclass[aspectratio=169, compress]{beamer}
% \usetheme{Boadilla}
% \usecolortheme{beaver}

\usetheme{Boadilla}
\usecolortheme{default}
\usepackage{beamerthemesplit}

\usefonttheme{serif}

\usepackage{lipsum}
\usepackage{graphicx,xcolor}
\usepackage{amsmath,amssymb,amsfonts}
\usepackage{tcolorbox}
\tcbuselibrary{theorems}

\usepackage[ruled, vlined]{algorithm2e}
\usepackage{listings}

\definecolor{codegreen}{rgb}{0,0.6,0}
\definecolor{codegray}{rgb}{0.5,0.5,0.5}
\definecolor{codepurple}{rgb}{0.58,0,0.82}
\definecolor{backcolour}{rgb}{1,1,1}

\lstdefinestyle{codestyle}{
    backgroundcolor=\color{backcolour},   
    commentstyle=\color{codegreen},
    keywordstyle=\color{magenta},
    numberstyle=\tiny\color{codegray},
    stringstyle=\color{codepurple},
    basicstyle=\ttfamily\footnotesize,
    breakatwhitespace=false,         
    breaklines=true,                 
    captionpos=b,                    
    keepspaces=true,                   
    numbersep=5pt,                  
    showspaces=false,                
    showstringspaces=false,
    showtabs=false,                  
    tabsize=4
}


\lstset{style=codestyle}

\newtcbtheorem[number within=section]{theo}{Theorem}%
{colback=green!5,colframe=green!35!black,fonttitle=\bfseries}{th}

\newtcbtheorem[number within=section]{lemm}{Lemma}%
{colback=yellow!30,colframe=red!75!black,fonttitle=\bfseries}{th}

\newtcbtheorem[number within=section]{hypothesis}{Hypothesis}%
{colback=red!5,colframe=red!35!black,fonttitle=\bfseries}{th}

\newtcbtheorem[number within=section]{define}{Definition}%
{colback=blue!5,colframe=blue!35!black,fonttitle=\bfseries}{th}


\title[CS6150 by theroyakash (2023) \hspace{0.5cm}\insertframenumber/\inserttotalframenumber]{Introduction to STL and Advanced STL}
\institute{\textsc{Indian Institute of Technology Madras}}
\logo{\includegraphics[height=0.75cm]{iitm.png}}
\author{\href{https://www.theroyakash.com}{theroyakash}}

\date{\today{}}


\begin{document}

\maketitle

\section{Introduction}
\begin{frame}{Introduction}
    \begin{itemize}
        \item Goal of the complexity theory is to understand computational difficulty of scientific problems.
    \end{itemize}
\end{frame}

\begin{frame}{Introduction}
    \begin{itemize}
        \item Goal of the complexity theory is to understand computational difficulty of engineering problems.
        \item So we've developed theory to classify problems according to their worst case behaviour. These classes are \textsc{P}, \textsc{NP} etc.
    \end{itemize}
\end{frame}

\begin{frame}{Introduction}
    \begin{itemize}
        \item Goal of the complexity theory is to understand computational difficulty of engineering problems.
        \item So we've developed theory to classify problems according to their worst case behaviour. These classes are \textsc{P}, \textsc{NP} etc.
        \item \textsc{P} class contains all the computational problems that in the worst case completes in polynomial time with respect to the size of the input.
    \end{itemize}
\end{frame}

\begin{frame}{Introduction}
    \begin{center}
        In our \textsc{CS6122} Course we've already seen that real world instances for few NP-Complete problems performs \textbf{good} in terms of running time.
    \end{center}
\end{frame}

\begin{frame}{Introduction}
    \begin{center}
        Thus we must develop theory that'll classify problems of their computational difficulty \textbf{with respect to real world performance} as well. Thus we develop smooth complexity theory.
    \end{center}
\end{frame}

\begin{frame}{Topics we'll look into}
    In this presentation we'll look into the following
    \begin{itemize}
        \item Basic Definitions and assumptions, $\textsf{Smoothed-P}$ Class.
            \begin{itemize}
                \item Model of smoothed analysis,
                \item Support of the distribution, notion of $N_{x, n}$.
                \item Concept of Family of Distribution
                \item Definition of smoothed polynomial running time \textbf{Definition 2.1},
                \item Definition of $\textsf{Smoothed-P}$
                \item \textbf{Theorem 2.3} \textit{An algorithm A has smoothed polynomial running time if and only if there is an} $\epsilon > 0$ \textit{and a polynomial} $p$ \textit{such that for all n, x, $\phi$ and t} $$\Pr_{y \sim D_{n, \phi, x}}[t_A(y; n, \phi) \geq t] \leq \frac{p(n)}{t^\epsilon} N_{n, x} \phi$$
            \end{itemize}
    \end{itemize}

\end{frame}

\begin{frame}{Topics - Continued}
    \begin{itemize}
        \item Heurisitic Schemes, error less heuristic schemes in $\textsf{Smoothed-P}$.
        \item Notion of Reduciblity, define $L_{ds}$, notion of completeness.
            \begin{itemize}
                \item Distributional problems
                \item Polynomial time smoothed reductions $\leq_{smoothed}$
            \end{itemize}
    \end{itemize}
\end{frame}


\begin{frame}{Topics - Continued}
    \begin{itemize}
        \item Tractability of problems like Binary Decision Problems,
        \item We'll briefly look into Smoothed Extension of $G_{n, p}$
        \item Concluding remarks
    \end{itemize}
\end{frame}

\section{Data Structure}
\begin{frame}
    \frametitle{Simple Greedy Heuristic for Packing Coloring}

    \begin{itemize}
        \item \textit{We'll start our algorithm to work specifically on trees first because it is easier to analyze and then we'll extend it to general graphs.} \pause[]
        \item \textit{In trees every odd layer we can color with a single color $1$ as every odd layer nodes are at distance more than $1$.}\pause[]
        \item \textit{Hence number of Node remains to be colored is significantly less than the total nodes ($n$). For example a complete $3$-ary tree we can color $75\%$ of the nodes with color $1$.}\pause[]
        \item \textit{We first see how algorithm working on a complete trees. Then we'll look into some of the optimizations we can do to improve the performance of the algorithm.}
    \end{itemize}

\end{frame}

\begin{frame}
    \frametitle{Simple Greedy Heuristic for Packing Coloring}

    \begin{algorithm}[H]\label{alg:1}
        \SetKwInOut{Input}{Input}
    
        \Input{Tree $T$}
        \BlankLine
        Compute Level order traversal of Tree $T$\;
        Color Every Odd layer nodes with \textbf{COLOR(1)}\;
        \texttt{level} $\gets d - 1$ ($d$ is the last level)\;
        \While{\texttt{level} $\geq 0$}{
            \texttt{maximum\_permissible\_color} = $n$\;
            \texttt{current\_color} = $2$\;
            \ForEach{\textsf{Node} in this \textsf{level}}{
                \While{\texttt{current\_color} $<$ \texttt{maximum\_permissible\_color}}{
                    Travel to every node within distance \texttt{( int ) current\_color} and check if there is any node
                    colored with color \texttt{current\_color}\;
                }
            }
        }
        \caption{\textsc{Basic Greedy Algorithm For Any tree}}
    \end{algorithm}

\end{frame}


\begin{frame}
    \frametitle{Simple Greedy Heuristic for Packing Coloring}

    \begin{algorithm}[H]
        \SetKwInOut{Input}{Input}
    
            Travel to every node within distance \texttt{( int ) current\_color} and check if there is any node
            colored with color \texttt{current\_color}\;

            \uIf{None of the node is colored with color \texttt{current\_color}} {
                Color this node with color \texttt{current\_color}\;
                \textbf{break from the loop, go to next node in level}\;
            }\Else{
                \texttt{current\_color} $\gets$ \texttt{current\_color} $+1$\;
            }
            \textsf{level} $\gets$ \textsf{level} $-1$\;
    
        \textbf{Output:} Output this coloring assignment.
    
        \caption{\textsc{Basic Greedy Algorithm For Any tree}}
    \end{algorithm}

\end{frame}

\begin{frame}
    \frametitle{Analysis of the Basic Algorithm}

    \textit{During the analysis we find that there are optimizations we can do to improve the run-time of our algorithm.}

\end{frame}


\begin{frame}
    \frametitle{Complexity Analysis}

    Our algorithm for each node $i \in (1, n)$ in the worst case visits all the $n$ node to find a color (from $1 \to n$). Hence worst case time complexity is $O(n^3)$.

\end{frame}

\begin{frame}
    \frametitle{Optimizations 1}

    We observe one simple fact, that for any complete tree, the maximum number of nodes at any level is present at the last level ($=x^d$, $x$ is the number of children and $d$ is the depth of the last level starting root from $0$).    

\end{frame}


\begin{frame}
    \frametitle{Optimizations 1}

    We are coloring every odd layer with color $1$. \pause[] 
    
    Instead of that if we color the last level and then every alternate level with color $1$ we'll color much more nodes with color $1$ and reduce the total number of colors used. Here is a simple example how this optimization saved thousands of colors.

\end{frame}

\begin{frame}
    \frametitle{Optimizations 1}

    \begin{table}[h]
        \centering
    \begin{tabular}{|l|l|r|l|}
    \hline
    \multicolumn{1}{|l|}{Nodes} & \multicolumn{1}{l|}{Layers} & \multicolumn{1}{l|}{Maximum Colors used} & \multicolumn{1}{l|}{Runtime} \\ \hline
    265720 & 12 & 20633 & 52m 32s 280ms \\ \hline
    265720 & 12 & 6890 & 4m 17s 31ms \\ \hline
    \end{tabular}
    \end{table}

    \pause
    \textit{This one simple optimization reduces the runtime by $92\%$}

\end{frame}



\begin{frame}
    \frametitle{Optimizations 2}

    Suppose we are at the moment trying to color node $u$. Our algorithm for each color $i \in (1, n)$ goes to distance $i$ from the node $u$ and checks if that color exists already or not in all the nodes sitting within distance $i$ from node $u$?

    \pause[]

    \textit{Lets see this step of the basic algorithm with an example.}

\end{frame}

\begin{frame}
    \frametitle{Optimizations 2}

    
    \begin{itemize}
        \item Suppose we are currently looking to color some node $u$ with color $d$.\pause[]
        \item We went to $d$ distance from node $u$ to find all the colors we find. Suppose we find color $d, d + 1, \dots, d+k$ are present in some of the nodes.\pause[]
        \item So we should not check this again for node $u$ with color $d + (1 \to k)$.\pause[]
        \item Hence we implement this modification to improve the runtime.
    \end{itemize}
\end{frame}


\begin{frame}
    \frametitle{Optimizations 2}
    \textit{We define a subroutine called \texttt{Check(u, d)}. This subroutine returns a set of colors present within distance $d$ for any node $u$.}
    \pause[]
    \begin{algorithm}[H]\label{alg:11}
        $\mathcal{C} \gets \phi$\;
    
        Visit all nodes within distance $d$ from node $u$ and collect all the colors into $\mathcal{C}$\;
    
        \textbf{return} Set $\mathcal{C}$\;
        
        \caption{\texttt{Check(Node u, Color d)}}
    \end{algorithm}


    \pause[]
    We can call this subroutine from the main coloring BFS call (we are coloring left to right, level by level). We start with the color $2$ and then we follow the following coloring strategy.
\end{frame}

\begin{frame}
    \frametitle{New coloring strategy}

    \begin{algorithm}[H]
        \For{Each node from last uncolored level, left to right} {
            $d_{\text{prev}}$ $\gets \phi$\;
            \For{Each Color $i$ from $2 \to n$} {
    
                \uIf{Color $i \in d_{\text{prev}}$}{
                It is not possible to color this node with color $i$ because we found color $i$ at distance less than $i$ in $d_{\text{prev}}$\;
                    Continue with color $i + 1$\;
                }
    
                \Else {
                    
                }
            }
        }
        \caption{\texttt{Updated Main Coloring Scheme}}
    \end{algorithm}

\end{frame}


\begin{frame}
    \frametitle{New coloring strategy}

    \begin{algorithm}[H]
        \For{Each node from last uncolored level, left to right} {
            $d_{\text{prev}}$ $\gets \phi$\;
            \For{Each Color $i$ from $2 \to n$} {
                \uIf{Color $i \in d_{\text{prev}}$}{

                }
                \Else{
                    $d_{\text{new}} = $ \texttt{Check(node, i, $d_{\text{prev}}$)}\;
    
                    \uIf{$i \notin d_{\text{new}}$}{
                        Color this node with color $i$\;
                        Break from this loop and start coloring next uncolored node\;
                    } \Else{
                        $d_{\text{prev}} = d_{\text{new}}$
                    }
                }
            }
        }
        \caption{\texttt{Updated Main Coloring Scheme}}
    \end{algorithm}
    
\end{frame}


\begin{frame}
    \frametitle{Optimization 3}

    
    \begin{itemize}
        \item This optimization comes from the observations of the structure of the complete trees.\pause
        \item Complete $x$-ary trees has a depth of $\log_{x} n$ with $n$ many nodes in them.\pause[]
        \item With the following optimization our algorithm time complexity will reduce from $O(n^3)$ down to $O(n d^2)$ for $x$-ary trees with $d$ diameter. This is a significant complexity improvement.
    \end{itemize}    

\end{frame}

\begin{frame}
    \frametitle{Optimization 3}

    
    \begin{itemize}
        \item Suppose $j$ is a color that has been used in the tree for the first time (during our run of the algorithm).\pause[] 
        \item If $j >$ the longest path in the tree, then color $j$ can never be used again.\pause[]
        \item Any color after $j$ that is $j + 1$ and so on will also not be possible to reuse.\pause[]
        \item So there is a upper bound on the number of color that are reusable. This depends on the longest path on the tree.
    \end{itemize}    

\end{frame}
\begin{frame}
    \frametitle{Sorting + Searching}

    Before introducing other data structure in the STL library,
    I'll show you some algorithms and iterator access on vector which is used often.


    \begin{itemize}
        \item Iterators
        \item Sorting
        \item Searching
    \end{itemize}

\end{frame}

\begin{frame}
    \frametitle{Iterators}

    
    \begin{itemize}
        \item Similar to pointers in C, C++ has Iterators \pause[]
        \item \texttt{vector<int>::iterator it = v.begin();} returns a \textit{Pointer} to the first element of vector, \pause[]
        \item Or you can use \texttt{auto it = v.begin();}\pause[]
        \item For example if you have an array $v = [10, 12, 13,14]$ then \texttt{*it} would return $10$,\pause[]
        \item Similar to pointer you can increase and decrease them, \texttt{it++;} and then \texttt{*it} would return $12$.
    \end{itemize}

\end{frame}

\begin{frame}
    \frametitle{Last Iterator}

    
    \begin{itemize}
        \item \texttt{vector<int>::iterator it = v.end();} points to a non-existent element sits after the last element.\pause[]
        \item Hence \texttt{*it} would dereference nothing when \texttt{it = v.end();}.
    \end{itemize}

\end{frame}

\begin{frame}
    \frametitle{Sorting}

    Sorting and searching is the most common thing you do on a vector.

    \lstinputlisting[language=C]{codes/9.cpp}

\end{frame}

\begin{frame}
    \frametitle{Custom Sorting}

    \textbf{What to do when you have a vector of custom data structure?}
    \pause[]
    
    For that we need to design custom comparators.

\end{frame}

\begin{frame}
    \frametitle{Custom Comparators Showcase}
    \lstinputlisting[language=C]{codes/10.cpp}
\end{frame}

\begin{frame}
    \frametitle{Custom Comparators Showcase II}
    \textbf{Using this you can define your own rule for sorting}, for example following shows
    how to sort in decreasing order.
    \pause[]

    \lstinputlisting[language=C]{codes/11.cpp}
\end{frame}


\begin{frame}
    \frametitle{Custom Comparators Showcase III}
    \textbf{Following is an example of sorting a custom class of data}.
    \pause[]
    
    \lstinputlisting[language=C]{codes/12.cpp}
\end{frame}

\begin{frame}
    \frametitle{Note}

    This comparator should return true for argument $(a, b)$ if and only if
    $a$ sits left of $b$ in the sorted array.

\end{frame}

\begin{frame}
    \frametitle{List container}

    
    \begin{itemize}
        \item A templated doubly linked list, this includes functions such as \texttt{push\_front} \texttt{push\_back}, \texttt{insert}, \texttt{erase}, etc.
    \end{itemize}

\end{frame}

\end{document}