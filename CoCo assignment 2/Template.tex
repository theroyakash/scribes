\documentclass[12pt,a4, onecolumn]{exam}
\usepackage{amsmath}
\usepackage{amssymb}
\usepackage{graphicx}
\graphicspath{ {./images/} }
\usepackage[lmargin=71pt, tmargin=1.2in]{geometry}  %For centering solution box
% \lhead{Leaft Header\\}
% \rhead{Right Header\\}
% \chead{\hline} % Un-comment to draw line below header
\thispagestyle{empty}   %For removing header/footer from page 1

\begin{document}

\begingroup  
    \centering
    \LARGE Computability and Complexity theory\\
    \LARGE Assignment \# 2\\[0.5em]
    \large \today\\[0.5em]
    \large Roy, Akash $\mid$ CS22M007\par
    \large M.Tech CS $\mid$ Indian Institute of Technology, Madras\par
\endgroup
\pointsdroppedatright   %Self-explanatory
\printanswers
\renewcommand{\solutiontitle}{\noindent\textbf{Ans:}\enspace}   %Replace "Ans:" with starting keyword in solution box

\begin{questions}

    \question Recall the definition of $\overline{HP}$ defined in the class. Let ${\sf HP}' = \Sigma^* \setminus {HP}$ (i.e. the set-theoretic compliment).  Show that $\overline{HP}$ is recursively enumerable if and only if ${\sf HP}'$ is recursively enumerable.
    
    \begin{solution}
        $\overline{\textbf{HP}}$ defined in the class was $\{M\#x \mid \text{ M does not halt on x}\}$.
        We can write the set-theoretic compliment $HP$ as $\overline{\textbf{HP}}$ defined in the class was $\{M\#x \mid \text{M does not halt on x}\} \bigcup A$
        where $A = \{M\#x \mid$ M is not a valid encoding of turing machine and x is some arbitrary string\}.

        Now from the question we need to show that
        $$
        \overline{\textbf{HP}} \in \textbf{REL} \iff \Sigma^* \setminus {HP} \in \textbf{REL}
        $$

        \textbf{Forward direction}: We get that $\overline{\textbf{HP}} \in \textbf{REL}$ so there exists a non total turing machine $\mathbb{M}$ accepting $\overline{\textbf{HP}}$. So for all the strings in $A = \{M\#x \mid$ M is not a valid encoding of turing machine and x is some arbitrary string\} we define a fixed turing machine $\hat{M}$ such that no matter what the input is $\hat{M}$ is always accepting. 
        We begin by constructing a new turing machine $M^{\alpha}$ that simulates $\mathbb{M}$ on x for input $M\#x$ where $M$ is valid and if some $M\#x \in A$ comes where M is not a valid encoding of a turing machine and x some random input we convert that to $\hat{M}$ and accept x.
        So then $\overline{\textbf{HP}} \bigcup \textbf{A}$ also becomes \textbf{REL}.
        $\implies \Sigma^* \setminus {HP} \in \textbf{REL}$

        \textbf{Backward direction} It is in the statement that $\Sigma^* \setminus {HP} \in \textbf{REL}$ and our class defined $\overline{\textbf{HP}}$ is a subset of $\Sigma^* \setminus {HP} \in \textbf{REL}$. Any subset of \textbf{REL} will always will be \textbf{REL}.

        Hence proved $\overline{\textbf{HP}} \in \textbf{REL} \iff \Sigma^* \setminus {HP} \in \textbf{REL}$.
    \end{solution}
    
    
    
    \question[Question - 2] Show that the  relation $\le_m$ is transitive, i.e., if $A\le_m B$, and $B\le_m C$, then $A\le_m C$.
    % \begin{parts}
    %     \part Part. a
    %     \part Hint: These can be nested further\droppoints
    % \end{parts}
    
    % \begin{solution}
    %         A2.
    %     \begin{parts}
    %         \part Ans. for (a)
    %         \part Solution for (b)
    %     \end{parts}
    % \end{solution}

    
    \begin{solution}
    It is given that $A\le_m B$ holds good, this implies there exists some function $f \colon \Sigma^* \to \Delta^*$ for which

    $$\forall x \in A \iff f(x) \in B$$

    and $f$ is computable.

    So a similar statement is also true for $B\le_m C$. So there exists some function $g \colon \Sigma^* \to \Delta^*$ for which

    $$\forall x_1 \in B \iff g(x_1) \in C$$
    and $g$ is computable.

    So now let's take an element $k$ from set \textbf{A}. For $k$ present in A there must be some element $j$ present in \textbf{B}, and for that element $j \in B$ there must exists an element $l \in C$. So for every element $k \in A$, there exists an element $l \in C$.

    Similar statement can be said for non-membership also.

    $$\forall x \notin A \iff f(x) \notin B \:\&\: \forall x_1 \notin B \iff g(x_1) \notin C$$
    
    The two statements up above means membership in A is many to one reducible to membership in C.

    So we can draw the conclusion that if $A \le_m B$ and $B \le_m C$ then $A \le_m C$.
    \end{solution}

    \question Show that if a language $A\subseteq \{0,1\}^*$ is recursive, then so is the concatenation $AA\stackrel{\triangle}{=}\{xy~|~ x,y \in A\}$.

    \begin{solution}
    This problem is equivalent to proving whether recursive languages are closed under operation concatenation?

    Given that $A \subseteq \{0,1\}^*$ is recursive means there exists a total Turing machine M accepting A. We construct a new Turing machine $\hat{M}$ in the following way
    \begin{itemize}
        \item For each input $y$ we do the followings, say size of input is $n$,
        \item \textbf{RUN 1:} We run $y[0] \:\&\: y[1 \to n]$ on Turing machine M. M either accepts or rejects $y[0]$ and $y[1 \to n]$. We define the behaviour of turing machine $\hat{M}$ the following way that if both $y[0]$ and $y[1 \to n]$ is rejected by total turing machine M, we go to \textbf{RUN: 2}. If either one of them is accepted and other is rejected that means the concatenation is not good we concatenated $x \in A$ with something $y \notin A$, so we reject. Accept if both is accepted means we've found an $xy$ such that $x \in A \:\&\: y \in A$, where $x=\text{input}[0] \:\&\: y = \text{input}[1 \to n-1]$
        \item \textbf{RUN 2:} We run $y[0 \to 1] \:\&\: y[2 \to n]$ on Turing machine M. M either accepts or rejects $y[0 \to 1] \:\&\: y[2 \to n]$. If both are rejected we go to \textbf{RUN 3}. Same otherwise, reject if one is accepted and other is rejected (\textbf{NOT} a valid concatenation). Accept if both is accepted.
        \item So on...
        \item At \textbf{RUN K} we check if $y[0 \to K] \:\&\: y[K+1 \to n]$ is accepted. Same as before if either is accepted and other is rejected then reject, if both of them is accepted accept and if both are rejected go to \textbf{RUN K + 1},
        \item We stop when we check $y[0 \to n-1] \:\&\: y[n]$. If these 2 are not accepted, if one is accepted and one is rejected we Reject and halt, if both are accepted we accept.
    \end{itemize}

    Language of this turing machine $\hat{M}$ is $\{xy \mid x \in A \:\&\: y \in A\}$ = $AA$. If we see for all $y \in \Sigma^*$ $\hat{M}$ is always halting. So $\hat{M}$ is total turing machine. So concatenation language $AA$ is recursive.
    
    \end{solution}

    \question Prove  (without using Rice's theorem) that the set \[\{ (M_1, M_2)~|~\mbox{ there is some input where both $M_1$  and $M_2 $ halt}\}\]is not recursive.

    \begin{solution}
        Say $D$ = \[\{ (M_1, M_2)~|~\mbox{ there is some input where both $M_1$  and $M_2 $ halt}\}\]
        
        We begin the proof by reduction from $H_P$. So there should be some computable function $f$ such that $H_p \leq_m D$

        We construct $\hat{M}$. $\hat{M}$ takes input $y \in \Sigma^*$. $\hat{M}$ halts on $y$ if and only if $M$ halts on $x$.
        
        So construction of $\hat{M}$ and Let $M$ be some turing machine,
        \begin{itemize}
            \item Take input $y \in \Sigma^*$
            \item Erase y
            \item Write x on input tape
            \item Run M on x
            \item If M halts on x, halt $\hat{M}$
        \end{itemize}

        M halts on $x$ implies $M\#x \in H_P$. Now for all input $y \in \Sigma^*$ $\hat{M}$ halts $y$ if M halts on $x$.
        When M does not halt on $x$ we don't accept anything so $L(\hat{M}) = \phi$

        So $M\#x \in H_P \iff$ $f(M\#x) \in D$
        
        $\implies$ y is some input where $\hat{M}, \hat{M}$ halts. $\implies$ $(\hat{M}, \hat{M}) \in D$.

        So $H_p \leq_m D$. So $D$ is not recursive, it is recursively enumerable.
    \end{solution}

    \question An unreachable state is a state which can never be reached from the start state on any input.  Show that the language
		\[ \{ M\# q~|~ q \mbox{ is an unreachable state  of $M$ on any input} \}\]
		is not recursive. (Your proof should NOT use Rice's theorem.)
  
    \begin{solution}
        We begin by constructing a new Turing machine $\hat{M}$. We know what are all its accepting states, rejecting states, say this is a non-total turing machine.
        
        We define a new language \textbf{REACHABLE} = \{M\#q $\mid$ q is reachable state of M on any input\}. Then our question becomes the language $\overline{\textbf{REACHABLE}}$.

        We assume that \textbf{REACHABLE} is decidable hence recursive. Let's construct one more Turing Machine $M_B$, it'll $\forall x \in \Sigma*$ will run $\hat{M}$ on x. We know what are the accepting states in turing machine $\hat{M}$. So running $\hat{M}$ on $x$ if we see that for input $x$ accepting states are not reachable (because we assumed REACHABLE is decidable) we halt. So $\forall x \in \Sigma*$ our turing machine $\hat{M}$ becomes a total turing machine which contradicts with our construction. Hence our assumption \textbf{REACHABLE} is decidable hence recursive is not true.

        \textbf{REACHABLE} is not recursive so it may be $\in$ REL or $\in$ Non-REL. (REL means recursively enumerable). So complement of \textbf{REACHABLE} which is our question will not be $\in$ REC. In particular if \textbf{REACHABLE} is in REL then $\overline{\textbf{REACHABLE}}$ must not be REL. But whatever is true $\overline{\textbf{REACHABLE}}$ is never recursive.
    \end{solution}
    
\end{questions}
\end{document}