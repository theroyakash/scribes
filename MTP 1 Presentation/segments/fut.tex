\begin{frame}
    \frametitle{Experimental Results}
    Here I'll show how the experimental results support our theoritical analysis.
\end{frame}

\begin{frame}
    \frametitle{Experimental Results}

    \begin{table}[h]
        \centering
        \begin{tabular}{|l|l|l|l|l|}
        \hline
        \# of Nodes & \# of Layers & X-ary tree & Total Colors & Time\\
        \hline
        13 & 3 & 3 & 4 & 0ms \\\hline
        40 & 4 & 3 & 7 & 0ms \\\hline
        121 & 5 & 3 & 11 & 2ms \\\hline
        364 & 6 & 3 & 19 & 6.11029 ms \\\hline
        1093 & 7 & 3 & 40 & 38ms \\\hline
        3280 & 8 & 3 & 98 & 63ms \\\hline
        9841 & 9 & 3 & 269 & 1s 101ms \\\hline
        29524 & 10 & 3 & 781 & 3s 845ms \\\hline
        88573 & 11 & 3 & 2309 & 45s 256ms \\\hline
        265720 & 12 & 3 & 6890 & 4m 17s 31ms \\ \hline
        7174453 & 15 & 3 & 185525 & 9 days 7 hours 29 min 9 sec\\
        \hline
        \end{tabular}
        \caption{Runtime, total color used for a complete three-ary tree.}
        \end{table}
\end{frame}

\begin{frame}
    \frametitle{Total Colors Used}

    All the experimental results show that the total number of colors used is less than $\frac{n}{40}$ where $n$ is the number of nodes in the tree. 
    
    \pause[]
    This is in line with our theoritical analysis.

\end{frame}

\begin{frame}
    \frametitle{Total Colors Used}

\begin{table}[h]
    \centering
    \begin{tabular}{|l|l|}
        \hline
        \textbf{Color number} & \textbf{Number of nodes} \\ \hline
        1                      & 5380840                   \\ \hline
        2, 3                      & 538084                    \\ \hline
        4                      & 177391                    \\ \hline
        5                      & 177390                    \\ \hline
        6, 7                   & 59058                     \\ \hline
        8, 9                   & 19684                     \\ \hline
        10, 11                 & 6561                      \\ \hline
    \end{tabular}
    \caption{Number of nodes colored with each color for a 15-layer complete tree.}
    \label{tab:colors}
    \end{table}
\end{frame}


\begin{frame}
    \frametitle{Total Colors Used}
    \begin{table}[h]
        \centering
        \begin{tabular}{|l|l|}
            \hline
            \textbf{Color number} & \textbf{Number of nodes} \\ \hline
            12, 13                 & 2187                      \\ \hline
            14, 15                 & 729                       \\ \hline
            16, 17                 & 243                       \\ \hline
            18, 19                 & 81                        \\ \hline
            20, 21                 & 27                        \\ \hline
            22, 23                 & 9                         \\ \hline
            24, 25                 & 3                         \\ \hline
            26 $\to$ 185525  & 1\\ \hline
        \end{tabular}
        \caption{Number of nodes colored with each color for a 15-layer complete tree.}
        \label{tab:colors}
    \end{table}
\end{frame}

\begin{frame}
    \frametitle{Experimental Results}

    These experimental results prove the following things

    \pause[]


    \begin{itemize}
        \item Color $2$ is used $\frac{1}{10}$th of the total number of nodes colored with color $1$.\pause[]
        \item Color $1$ is used for the $\frac{3}{4}^{\text{th}}$ of the total number of nodes.\pause[]
        \item Pair-wise colors are used for the same amount of nodes in the tree.\pause[]
        \item After $26$ which is some less than $2*d = 2*15 = 30$, all the colors are used for only once.
    \end{itemize}

\end{frame}

\begin{frame}
    \frametitle{Conclusion}

    This completes our presentation on the approximation algorithm for packing coloring on trees.

    \pause[]

    \textit{In future we'll look for the following things:}\pause[]
    
    \begin{itemize}
        \item We analysed our algorithm performance on a complete three-ary tree and some trees with randomly delete branch. We need to analyse the performance for any $d$-degree bounded tree and graphs. To do this one approach we thought of is to design an algorithm that'll find a suitable root such that most of the nodes are equidistant from this root. Suitable root must decrease the number of colors to be used by our algorithm.\pause[]
        \item To test the effectiveness of the algorithm we need to come up with a random-graph generation scheme. Through which we can generate graphs at random with certain properties and review our algorithm performance.\pause[]
    \end{itemize}
    
\end{frame}

\begin{frame}
    \frametitle{Future}


    \begin{itemize}
        \item We need to come up with a randomized graph generation scheme that'll generate worst case graphs to color for our algorithm. This will generate the worst case graphs, that'll cost very significant amount of colors to color according to our coloring strategy and some fix for those graphs. We also need to check the performance of our algorithm on randomly chosen graph from a fixed distribution.
    \end{itemize}

\end{frame}

\begin{frame}
    \frametitle{Thank You}

    \begin{center}
        \Huge{Thank You}
    \end{center}

\end{frame}