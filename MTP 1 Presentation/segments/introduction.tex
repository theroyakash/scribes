\begin{frame}
    \frametitle{Introduction to Packing Coloring}
    
    We start with a few definitions. First we define what is graph coloring.\pause

    Let $G= (V,E)$ be an undirected graph.\pause
\end{frame}


\begin{frame}
    \frametitle{Graph Coloring}

    \begin{define}{Vertex Coloring}{}
        A vertex $k$ coloring of $G$ is a map $f: V \to \{1,\ldots, k\}$. A coloring $f$ is said to be proper, if for every edge $(u,v)\in E$, $f(u) \neq f(v)$. The chromatic number of a graph is the minimum value of $k$ such that $G$ has a proper $k$ colouring.
    \end{define}

\end{frame}

\begin{frame}
    \frametitle{Graph Packing Coloring}


    \begin{itemize}
        \item Given the centrality of graph colouring to graph theory and computer science, there have been several variants of colouring problems on graphs with some additional conditions imposed.\pause
        \item One such variant is the packing coloring problem.\pause
        \item List coloring, path coloring, repetition free colouring are some of the other prominent examples.
    \end{itemize}
\end{frame}

\begin{frame}
    \frametitle{Graph Packing Coloring}
    \textit{We  begin with a definition of the  graph packing-coloring problem.}
\end{frame}

\begin{frame}
    \frametitle{Graph Packing Coloring}

    \begin{define}{$S$-Packing Coloring and Packing Coloring}{}
        \textit{Suppose $S = (a_i)_{i \in \left[1 \to \infty\right)}$ is a increasing sequence of integers, then $S$ packing coloring of the graph is partition on the vertex set $V(G)$ into sets $V_1, V_2, V_3 \dots$ such that for every pair $(x, y) \in V_k$ is at a distance more than $a_k$. If $a_i = i$ for every $i \in \left[1 \to \infty\right)$, then we call the problem packing coloring.}
    \end{define}

\end{frame}


\begin{frame}
    \frametitle{Graph Packing Coloring}

    \begin{define}{$S$-Packing Chromatic Number}{}
        \textit{If there exists an integer $k$ such that $V(G) = V_1, V_2, V_3 \dots V_k$, each $V_i$ is a vertex-partition, then this partition is called $S$-packing, $k$ coloring, and minimum of such $k$ is the $S$-Packing Chromatic Number.}
    \end{define}

\end{frame}

\begin{frame}
    \frametitle{Difference between Packing Coloring and Graph Coloring}

    
    \begin{itemize}
        \item There are several approximation algorithms for graph coloring, but there aren't any approximation algorithms for packing coloring.\pause
        \item Any graph is three-colorable is a \textsf{NP}-complete problem. So there are reasons to develop an approximation scheme for graph coloring.\pause
        \item We also know from early on that the decision version of the packing coloring is a \textsf{NP} complete problem.\pause
        \item The decision version in the form of \textit{A graph G and a positive integer K, does G have a packing-K coloring}, is \textsf{NP}-complete for \textit{k = 4} even when restricted to \textit{planner} graphs.
    \end{itemize}

\end{frame}

\begin{frame}
    \frametitle{Packing Coloring on Trees}

    \textit{Packing coloring is also \textsf{NP}-Hard for the case of trees (which are acyclic undirected unweighted graphs).}    

    \pause
    It is one thing to compute the decision version of the packing coloring problem for which we don't have any efficient algorithm. It is equally or more difficult to have a fast algorithm for getting hold of an valid packing coloring assignment.

\end{frame}

\begin{frame}
    \frametitle{Packing Coloring on Trees}

    \textit{In this presentation I'll show an algorithm that gets us a valid packing coloring assignment in polynomial time. \pause However the algorithm do not get us the minimum number of colors that is the packing chromatic number. It is \textbf{an approximation} of the actual packing chromatic number.}

\end{frame}

\begin{frame}
    \frametitle{Simple Algorithm for Packing Coloring}

    \textit{We are to find a valid assignment of packing coloring to the vertices of the graph. \pause Most straight forward algorithm we can think of is simply assign some color, and backtrack and re-color in case of conflicts.}

\end{frame}