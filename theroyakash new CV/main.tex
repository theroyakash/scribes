% This work may be distributed and/or modified under the
% conditions of the LaTeX Project Public License version 1.3c,
% available at http://www.latex-project.org/lppl/.

% FOR VSCODE LATEX-COMPILER ARGUMENTS
%!TEX program = lualatex
% REMOVE THIS OTHERWISE

\documentclass[10pt, a4paper]{moderncv}
% possible options include font size ('10pt', '11pt' and '12pt'), paper size ('a4paper', 'letterpaper', 'a5paper', 'legalpaper', 'executivepaper' and 'landscape') and font family ('sans' and 'roman')

% moderncv themes
\moderncvstyle{banking}                            % style options are 'casual' (default), 'classic', 'oldstylef' and 'banking'
\moderncvcolor{red}                                % color options 'blue' (default), 'orange', 'green', 'red', 'purple', 'grey' and 'black'
% \renewcommand{\familydefault}{\rmdefault}         % to set the default font; use '\sfdefault' for the default sans serif font, '\rmdefault' for the default roman one, or any tex font name
%\nopagenumbers{}                                  % uncomment to suppress automatic page numbering for CVs longer than one page

\usepackage{fontspec}
\setmainfont{CMUSerif}
% character encoding
\usepackage[utf8]{inputenc}                       % if you are not using xelatex ou lualatex, replace by the encoding you are using
%\usepackage{CJKutf8}                              % if you need to use CJK to typeset your resume in Chinese, Japanese or Korean
% adjust the page margins
\usepackage[scale=0.9]{geometry}
%\setlength{\hintscolumnwidth}{3cm}                % if you want to change the width of the column with the dates
%\setlength{\makecvtitlenamewidth}{10cm}           % for the 'classic' style, if you want to force the width allocated to your name and avoid line breaks. be careful though, the length is normally calculated to avoid any overlap with your personal info; use this at your own typographical risks...


% personal data
\name{Roy}{Akash}
\title{CS22M007}
\insti{M.Tech CS $\mid$ Indian Institute of technology, Madras}
% \extrainfo{Roll Number: CS22M007}
% optional, remove / comment the line if not wanted; the "postcode city" and and "country" arguments can be omitted or provided empty
% \phone[mobile]{+91-9163611726}                   % optional, remove / comment the line if not wanted
%\phone[fixed]{+2~(345)~678~901}                    % optional, remove / comment the line if not wanted
%\phone[fax]{+3~(456)~789~012}                      % optional, remove / comment the line if not wanted
% \email{hey@theroyakash.com}                               % optional, remove / comment the line if not wanted
\homepage{www.theroyakash.com}                         % optional, remove / comment the line if not wanted

\extrainfo{\homepagesymbol \httplink{www.github.com/theroyakash}}
                % optional, remove / comment the line if not wanted
                % optional, remove / comment the line if not wanted
\photo[64pt][0.4pt]{iitm.png}                       % optional, remove / comment the line if not wanted; '64pt' is the height the picture must be resized to, 0.4pt is the thickness of the frame around it (put it to 0pt for no frame) and 'picture' is the name of the picture file
%\quote{Love thy journey, never thy destination!}                           
%----------------------------------------------------------------------------------
%            content
%----------------------------------------------------------------------------------
\begin{document}
%\begin{CJK*}{UTF8}{gbsn}                          % to typeset your resume in Chinese using CJK
%-----       resume       ---------------------------------------------------------
%\makecvtitle
\hypersetup{
    linkcolor=blue,
    filecolor=magenta,      
    urlcolor=cyan,
}
\urlstyle{same}

\noindent
\begin{minipage}{.78\textwidth}
 \makecvtitle
\end{minipage}%
\begin{minipage}{.30\textwidth}
  \centering
  \includegraphics[height=3cm]{iitm.png}
\end{minipage}
\vspace{-1em}

\section{Education}
\vspace{0.3em}
\begin{minipage}{\textwidth}
\centering
\setlength{\tabcolsep}{10pt}
\begin{tabular}{|p{3cm}|p{8cm}|p{2cm}|p{3cm}|}
\hline
\textbf{Program} & \textbf{Institution} & \textbf{\%/CGPA} & \textbf{Year} \\ \hline
\textbf{M.Tech in CS} & Indian Institute of Technology, Madras & 8.58/10 & 2022-2024 \\ \hline
\textbf{B.Tech in CS} & Government College of Engineering and Ceramic Technology & 8.94/10 & 2018-2022 \\ \hline
\textbf{HS (Class 12)} & Barasat Mahatma Gandhi Memorial High School & 78.60\% & 2018 \\\hline
\textbf{Class 10} & Barasat Mahatma Gandhi Memorial High School & 88.71\% & 2016 \\
\hline
\end{tabular}

\end{minipage}

\vspace{1em}
\section{Work Experience}
\cventry{}{India, Hyderabad Office, Joining June 2024}{Software Engineer at Microsoft India}{June 2024 - Present}{}{}


\cventry{}{India, Bangalore Office, Registration and Core Services Team}{SDE intern at Amazon India}{May 2023 - July 2023}{}{}
\cvlistitem{Received an inclined offer based on the performance during the Internship.}
\cvlistitem{Internship with the Registration and Core Services Team at Amazon Bangalore. Implemented a \textbf{redirection feature} which is currently deployed to \textbf{production environment} on the \textbf{Amazon Worldwide Seller Central}, which automatically redirects seller from Add new bank account page to the account verification page. This implementation improves customer experience and reduces time and effort in verifying bank accounts. The framework did not inherently support redirection. Using \textbf{WebLabs (selective deployment)}, we \textbf{eventually deployed} redirection feature to sellers.}
\cvlistitem{In our implementation, we found one edge case for which our redirection feature was incomplete. To solve this problem, I designed and implemented \textbf{foundational API changes} along with suggestions from SDE-I and SDE-III. These code changes migrated logic from service orchestration package to service management package, which then calls service orchestration package based on certain parameters. This movement of orchestration logic into management service is a foundational code change that is a stepping stone toward team vision.}

\cventry{}{Introduction to programming, Object Oriented Programming, C++ Advanced Programming}{Teaching Assistant at CS Department, IIT Madras}{July 2022 - May 2024}{}{}
\cvlistitem{Was a lead teaching assistant in CS1100 managing 70+ teaching assistants and 350+ students who took the course. Was responsible for smooth operation of the course to the completion and was responsible for \textbf{question design}, conducting exams, doubt clearing and leading the exam evaluation done by the other TAs.}

\cvlistitem{Taught various computer science courses at IIT Madras, including a lead teaching assistantship and an instructor position in the \textbf{CS6150 Advanced Programming Lab in C++} course for incoming CS Masters students, \textbf{CS5410 Security in Cyber Physical System} for the under-graduate and graduate students, \textbf{CS2810 Object Oriented Design Lab} for second-year CS B.Tech students, and \textbf{CS1100 Introduction to Programming} for incoming B.Tech students of various departments at IITM coming through JEE Advanced. Responsibilities encompassed designing questions, \textbf{conducting classes}, \textbf{teaching}, \textbf{automated test-case generation}, and verifying student-submitted code.}


\section{Publications}
% \cventry{IIT Madras}{During M.Tech}{Distributed Systems Design, a foundational book}{Spring 2022}{Book Coming Soon}{}
% \cvlistitem{Building distributed systems is difficult, often challenging. To get started with system designing, I wrote this book when I was starting to learn distributed systems from scratch}
% \cvlistitem{This book introduces people to the concept of distributed systems, tools, and algorithms that one needs to build those systems, helping them switch from \textbf{monolith} to \textbf{microservices} if needed in no time.}

\vspace{2mm}
\cventry{}{Over 400+ five star ratings}{Python basics to get started book}{Spring 2020}{Available on Google Play}{}
\cvlistitem{Python Basics to get started is an introductory Python book for first-year students to start with Python programming. With over 20,000 copies sold in the last three years, my Python introduction book is widely used by my juniors from B.Tech college and received over 480+ five-star ratings.}

% technical skills
\section{Technical Skills}
\cvlistdoubleitem{
{\textbf{Languages}}: C, C++, Python, Java, GoLang}{
{\textbf{Web}}: Python-Flask, Springboot}

\cvlistdoubleitem{
{\textbf{Database Systems}}: MySQL PostgreSQL, NoSQL}{
{\textbf{Systems}}: Docker, Redis, CI/CD}

% \cvlistitem{{\textbf{Computer Science Fundamentals}:} {Operating Systems, Computer Networks, Database Management Systems, Theory of Computation, Randomized algorithms, Design and analysis of algorithms, Probabilistic and Smoothed analysis of algorithms.}}


% KEY Projects

\section{Key Research and Technical Projects}


\cventry{Indian Institute of Technology, Madras}{Theoretical Computer Science, Graph Theory, Graph Coloring}
{Master's Research Project in Theoretical Computer Science}{July 2023 - May 2024}{}{}
\cvlistitem{Researcher in the \textbf{Theoretical CS group, IIT-M}, collaborating with \textbf{Dr. B. V. Raghavendra Rao.}}
\cvlistitem{Broadly studying the complexity of graph packing coloring problems. Designing, architecting, and implementing several schemes for packing coloring in degree-bounded trees and other special graphs.}


\cventry{Indian Institute of Technology, Madras}{GoLang, Python, Large Scale Distributed systems design, Database technology}
{Large Scale Distributed Notification System}{June 2023 - August 2023}{}{}
\cvlistitem{This notification service implementation can scale up to millions of customers to reliably deliver notifications. The primary purpose of such a service is to provide central or local government or police the ability to deliver disaster notifications to millions of residents.}

\cvlistitem{Built on Golang and Python, the main notification service is divided into four services \textit{RepositoryService}, \textit{AuthorizationService}, \textit{PersistentStoreService}, \textit{Publisher} and \textit{Subscriber}, each runs in their docker container, and are independently scalable. System Design, HLD, and other documentation can be found online.}


\cventry{Indian Institute of Technology, Madras}{Tailwind CSS, Vanilla JavaScript, ElectronJS}
{\textit{AmbientBuddy}}{June 2023 to July 2023}{}{}
\cvlistitem{Beautifully designed AmbientBuddy is a macOS, Windows-based meditation app that has 30+ ambient sounds built into it. AmbientBuddy has the best design among all available ambient apps and works offline. AmbientBuddy is designed to work offline so that users don't require an active connection to listen to the audio.}
\cvlistitem{Audios are also customizable; users can mix multiple sounds and re-adjust volumes of multiple audio to listen to a calming experience. AmbientBuddy has a beautifully designed landing page, and the code is open-sourced.}

% \cventry{Indian Institute of Technology, Madras}{Bootstrap CSS, Python, Flask, Sessions, SQLite3}
% {\textit{Password Manager}}{March 2022 to July 2022}{}{}

% \cvlistitem{Web-based locally-runnable password manager, that stores unlimited passwords in a local SQLite3 database,}
% \cvlistitem{Users are supposed to create a local account with a master password to manage and store passwords under that account. Data is stored in the user's personal computer, so there is no chance of data leaks. Users also have the option to self-host on any PaaS platform.}
% \cvlistitem{Users have the option to create multiple local profiles and store separate login information in separate profiles, and users also have the option to generate long hard to guess, random passwords that they can use in any services they want.}

\cventry{Indian Institute of Technology, Madras}{Free interview preparation platform}
{algorithms.theroyakash.com}{Feb 2022 to Today}{}{}
\cvlistitem{Used by 650+ users from all over the world, these are materials for coding interviews, has over 150+ interview questions with detailed explanations on several data structures and algorithm concepts. It is a starting point for many bachelor and master in computer science students to start their interview preparations.}

\vspace{1mm}

\cventry{Government College of Engineering and Ceramic Technology}{Python Package with PIP}
{\textit{AKDSFramework}}{Nov 2020 - Sep 2021}{}{}
\cvlistitem{Pure pythonic, efficient implementations for all popular data structures and algorithms. AKDSFramework provides one-line benchmarking tools, an easy-to-use one-of-a-kind Big O analyzer, and a caching decorator.}
\cvlistitem{AKDSFramework has API to draw graphs from an AKDSFramework graph object (using mermaid-js API in Python, which I contributed to the mermaid project). All run-able functions and modules are tested with Python unit tests using GitHub actions. Framework documentation and open source code are available online.}

% \cventry{Indian Institute of Technology, Madras}{Python}
% {{\href{https://theroyakash.notion.site/BOT-037afbdb4cd1417db23573e9434ef06b}{\textcolor{blue}{Multipurpose discord bot}}}}{April 2021 - Currently}{}{}
% \cvlistitem{This Python-based fun discord bot is a multipurpose discord bot consisting of many features, including getting memes from subreddits, getting a random non-existent AI-generated image (calling APIs), searching music on YouTube (using Google APIs), and several other image operations, getting news (by parsing Google News XML).}
% \cvlistitem{The discord bot also has a wine recommendation system that recommends wines from user inputs matching with the description in the database using \textbf{fuzzzy search}. Data is stored in a \textbf{SQLite} database.}

% Course Work

\section{Other Projects}

\cventry{Government College of Engineering and Ceramic technology}{Independent Research}{\texttt{AKHCRNet}}{April 2020 - October 2020}{}{}
\cvlistitem{\texttt{AKHCRNet} is a state-of-the-art deep neural architectural solution for handwritten character recognition for Bengali alphabets. Gained eight citations, including researchers from IITs.}

\cventry{Government College of Engineering and Ceramic Technology}{Python, GitHub Actions Cron Jobs, RestAPI}{@sosleafy News Bot}{Jun 2021}{}{}
\cvlistitem{@sosleafy is a Twitter account and a telegram group where every 3 hours, a bot posts fresh news content run via GitHub actions. Using \textbf{Github actions} and \textbf{Cron jobs}, the script takes new news content from the web and puts it in a GitHub repository. Then once every three hours, another \textbf{cron job} is scheduled to post new news content on Twitter and Telegram, setting the GitHub repository as the news source. In it's operation \textbf{@sosleafy} posted over \textbf{11K} tweets on twitter and currently shut down.}

\cventry{Government College of Engineering and Ceramic Technology}{Python installable package with Pip}{placeholderfile in \texttt{pip} package manager}{October 2020}{}{}
\cvlistitem{\texttt{placeholderfile} is a random filename and UUID generator installable via pip.}

\section{Course Work}
\cventry{Indian Institute of Technology Madras}{(Core and electives)}{Key Courses in M.Tech}{July 2022-May 2024}{}{}
\cvlistitem{\textit{Courses}: Advanced data structures and algorithms, Randomized Algorithms, Computability, and Complexity theory, Advanced Programming Lab, Probabilistic and Smoothed Analysis of Algorithms, Security in Cyber Physical Systems, Digital System testing and testable design, Theory and application of Ontologies.}

\cventry{Government College of Engineering and Ceramic technology}{(Core and electives)}{Key Courses in B.Tech}{August 2018-April 2022}{}{}
\cvlistitem{\textit{Courses}: Theory of computation, Discrete Maths, Computer Networks, Compiler Design, Design and analysis of Algorithms, Data Structures, Object Oriented design, Digital Logic, Computer Organizations, Operating Systems, Database management systems, Machine learning, Deep learning.}

% \section{Positions of Responsibility}
% \begin{itemize}
% \item \textit{Teaching Assistant} for Computer Science Department, Indian Institute of Technology, Madras (2022-2024).
% \end{itemize}


\section{Achievements}

\cvlistitem {Successfully qualified 
\textit{GATE CS (2022) with over 98.5} percentile,}

\cvlistitem {Became Course topper and Elite + Silver Medal in NPTEL Cloud computing course conducted by IIT Kharagpur,}

\cvlistitem {Got 70\% in Theory of computation NPTEL Course conducted by IIT Kanpur.}

%Languages (Spoken/known): Assamese, Bengali, English, Garo, Hindi, Nepali. 

% \section{Others}
% \cvlistdoubleitem{Hobbies: Video Making, Photography, Apps making}{Languages: Bengali, Hindi, English.}


% \section{Declaration}
% {I do hereby declare that all the details furnished above are true to the best of my knowledge and belief.}

% \vspace{4mm}
% {Last updated: 2023 March \hspace{35em} (Akash Roy)}
\end{document}
