\documentclass[12pt, a4paper, onecolumn]{exam}
\usepackage{amsmath}
\usepackage{amssymb}
\usepackage{graphicx}
\usepackage{pdfpages}
\usepackage[linesnumbered,ruled]{algorithm2e}
\graphicspath{ {./images/} }
\usepackage[lmargin=71pt, tmargin=1.2in]{geometry}  %For centering solution box
% \lhead{Leaft Header\\}
% \rhead{Right Header\\}
% \chead{\hline} % Un-comment to draw line below header
\thispagestyle{empty}   %For removing header/footer from page 1

\usepackage{tcolorbox}
\tcbuselibrary{theorems}

\newtcbtheorem[number within=section]{theo}{Theorem}%
{colback=green!5,colframe=green!35!black,fonttitle=\bfseries}{th}

\newtcbtheorem[number within=section]{hypothesis}{Hypothesis}%
{colback=red!5,colframe=red!35!black,fonttitle=\bfseries}{th}

\newtcbtheorem[number within=section]{define}{Definition}%
{colback=blue!5,colframe=blue!35!black,fonttitle=\bfseries}{th}

\newcommand{\RP}{\ensuremath{\mathsf{RP}}}
\newcommand{\expect}[1]{\ensuremath{\mathbb{E}[#1]}}
\newcommand{\dx}{\mathrm{d}x}
\newcommand{\NP}{\ensuremath{\mathsf{NP}}}
\newcommand{\co}{\ensuremath{\mathsf{co}\mbox{-}}}
\newcommand{\NL}{\ensuremath{\mathsf{NL}}}

\begin{document}

\hrule
\vspace{1em}

\begingroup
\centering
\LARGE Probabilistic and Smoothed Analysis of Algorithms\\
\LARGE Assignment \# 3\\[0.5em]
\large \today\\[0.5em]
\large \textsc{Roy, Akash} $\mid$ CS22M007\par
\large \textsc{M.Tech CS} $\mid$ \textsc{Indian Institute of Technology, Madras}\par
\endgroup
\pointsdroppedatright %Self-explanatory
\printanswers
\renewcommand{\solutiontitle}{\noindent\textbf{Ans:}\enspace}   %Replace "Ans:" with starting keyword in solution box

\vspace{1em}
\hrule
\vspace{0.2em}

\begin{center}
	\textsc{Answering any} $5$.
\end{center}

\vspace{0.2em}
\hrule
\vspace{1em}

\begin{questions}

	\question[10]  This exercise is to demonstrate the limitations of considering expected running time of an algorithm as a useful measure. For any $n>0$, describe  a function $f:\{0,1\}^n \to \mathbb{N}$ such that
	\begin{itemize}
		\item $\mathbb{E}[f] = \mathbb{E}_{x\in \{0,1\}^n}[f(x)] = n^c$ for some constant $c$ and
		\item ${\sf var}[f] = E[f^2] - (E[f])^2 = \Omega(2^n)$.
	\end{itemize}
	I.e., there can be performance measures $f$ which is polynomial in expectation, but variance being exponential. Give formal justification for your answer (i.e., computation of expectation and variance for the function $f$ constructed).  Further, write your views on why the function $f$ that you have constructed may/maynot be a good performance measure for any algorithm in practice.

	\question[10]  Compute the Fourier expansion of the following Boolean functions. (Note the functions are described in the Boolean/Fourier domain, you need to convert the function to Fourier domain first.)
	\begin{parts}
		\part[2] Majority function $\text{MAJ}_3 \{ -1,1\}^n \to \{-1,1\}$,
		$\text{MAJ}_3(x_1,x_2,x_3) = {\sf sgn}(x_1+x_2+x_3)$.

		\begin{solution}
			$\text{MAJ}_3$ computes Majority of three boolean variables $x_1$ and $x_2$ and $x_3$.

			We can plot the truth table for the $\text{MAJ}_3$ function which is the
			following,

			\vspace{0.7em}

			\begin{tabular}{|r|r|r|r|}
				\hline
				$x_1$ & $x_2$ & $x_3$ & $\text{Maj}_3(x_1, x_2, x_3)$ \\ \hline
				$1$   & $1$   & $1$   & $1$                           \\ \hline
				$1$   & $1$   & $-1$  & $1$                           \\ \hline
				$1$   & $-1$  & $1$   & $1$                           \\ \hline
				$-1$  & $1$   & $1$   & $1$                           \\ \hline
				$-1$  & $1$   & $-1$  & $-1$                          \\ \hline
				$-1$  & $-1$  & $1$   & $-1$                          \\ \hline
				$1$   & $-1$  & $-1$  & $-1$                          \\ \hline
				$-1$  & $-1$  & $-1$  & $-1$                          \\ \hline
			\end{tabular}

			Following this truth table we can arrive at the equation for $\text{Maj}_3$ via
			interpolation. For each point $(x_1, x_2, x_3)$ we need to cook a polynomal
			that is $1$ at $(x_1, x_2, x_3)$ and $0$ everywhere else.

			Such polynomal is the following. As $x_i \in \{-1,1\}$

			\begin{align*}
				\left( \frac{1 + a_1 x_1}{2} \cdot \frac{1 + a_2 x_2}{2} \cdot \frac{1 + a_3 x_3}{2} \right) =
				\begin{cases}
					1 \textsf{ if } (a_i) = \textsf{sgn}(x_i) \textsf{ and } x_i  = |x_i| \\
					0 \textsf{ otherwise}
				\end{cases}
			\end{align*}

			\begin{align*}
				\text{Maj}_3(x_1, x_2, x_3) =
				 & (+1) \left( \frac{1 + x_1}{2} \cdot \frac{1 + x_2}{2} \cdot \frac{1 + x_3}{2} \right) + \\
				 & (+1) \left( \frac{1 + x_1}{2} \cdot \frac{1 + x_2}{2} \cdot \frac{1 - x_3}{2} \right) + \\
				 & (+1) \left( \frac{1 + x_1}{2} \cdot \frac{1 - x_2}{2} \cdot \frac{1 + x_3}{2} \right) + \\
				 & (+1) \left( \frac{1 - x_1}{2} \cdot \frac{1 + x_2}{2} \cdot \frac{1 + x_3}{2} \right) + \\
				 & (-1) \left( \frac{1 - x_1}{2} \cdot \frac{1 + x_2}{2} \cdot \frac{1 - x_3}{2} \right) + \\
				 & (-1) \left( \frac{1 - x_1}{2} \cdot \frac{1 - x_2}{2} \cdot \frac{1 + x_3}{2} \right) + \\
				 & (-1) \left( \frac{1 + x_1}{2} \cdot \frac{1 - x_2}{2} \cdot \frac{1 - x_3}{2} \right) + \\
				 & (-1) \left( \frac{1 - x_1}{2} \cdot \frac{1 - x_2}{2} \cdot \frac{1 - x_3}{2} \right)
			\end{align*}

			Simplifying this will result into the following equation

			\begin{align*}
				\textsf{Maj}_3(x_1, x_2, x_3) = \frac{1}{2} x_1 + \frac{1}{2} x_2 + \frac{1}{2} x_3 - \frac{1}{2} x_1 \cdot x_2 \cdot x_3
			\end{align*}

		\end{solution}

		\part[3] $NAE_n:\{-1,1\}^n \to \mathbb{R}$, $NAE_n(x) =1$ if not all of the input
		bits are equal, and $0$ otherwise. Compute the Fourier coefficients for $n=4$.

		\begin{solution}
			Using interpolation technique we can get the truth table based representation of the function
			which is the following

			\begin{align*}
				\text{NAE}_4(x_1, x_2, x_3, x_4) &= \frac{1}{16}(1-x_1)(1-x_2)(1-x_3)(1-x_4) \cdot 0 \\ 
				&+\frac{1}{16}(1-x_1)(1-x_2)(1+x_3)(1-x_4) \cdot 1\\ 
				&+\frac{1}{16}(1-x_1)(1-x_2)(1+x_3)(1+x_4) \cdot 1\\ 
				&+\frac{1}{16}(1-x_1)(1+x_2)(1-x_3)(1-x_4) \cdot 1\\ 
				&+\frac{1}{16}(1-x_1)(1+x_2)(1-x_3)(1+x_4) \cdot 1\\ 
				&+\frac{1}{16}(1-x_1)(1+x_2)(1+x_3)(1-x_4) \cdot 1\\ 
				&+\frac{1}{16}(1-x_1)(1+x_2)(1+x_3)(1+x_4) \cdot 1\\ 
				&+\frac{1}{16}(1+x_1)(1-x_2)(1-x_3)(1-x_4) \cdot 1\\ 
				&+\frac{1}{16}(1+x_1)(1-x_2)(1-x_3)(1+x_4) \cdot 1\\ 
				&+\frac{1}{16}(1+x_1)(1-x_2)(1+x_3)(1-x_4) \cdot 1\\ 
				&+\frac{1}{16}(1+x_1)(1-x_2)(1+x_3)(1+x_4) \cdot 1\\ 
				&+\frac{1}{16}(1+x_1)(1+x_2)(1-x_3)(1-x_4) \cdot 1\\ 
				&+\frac{1}{16}(1+x_1)(1+x_2)(1-x_3)(1+x_4) \cdot 1\\ 
				&+\frac{1}{16}(1+x_1)(1+x_2)(1+x_3)(1-x_4) \cdot 1\\ 
				&+\frac{1}{16}(1+x_1)(1+x_2)(1+x_3)(1+x_4) \cdot 0\\
			\end{align*}

		\end{solution}

		\part[5] The inner product function: $IP_{2n}:\{-1,1\}^n \times \{-1,1\}^n \to
			\{-1,1\}$, given by $IP_{2n}(x,y) = {\sf sgn}(\sum_{i=1}^n x_iy_i) $. First
		write down the Fourier coefficients for the case of $n=2$ (3 points) and then
		obtain a general formula (2 points). No proof needed for the general formula.

		\begin{solution}

		\end{solution}

	\end{parts}

	\question[10] Exercise 1.8 in the book Analysis of Boolean Functions by Ryan O'Donell. Point split: 4+2+4.

	The (Boolean) dual of $f:\{-1,1\}^n\to \mathbb{R}$ is the function $f^{t}$
	defined by $f^{t} = - f(-x)$. The function $f$ is said to be odd if it equals
	its dual, equivalently if $f(-x) = -f(x) \: \forall x$. The function $f$ is
	said to be even if $f(-x) = f(x) \: \forall x$. Given any function
	$f:\{-1,1\}^n \to \mathbb{R}$ its odd part is the function $f^{\text{odd}} =
		\frac{f(x) - f(-x)}{2}$ and its even part is the function $f^{\text{even}} =
		\frac{f(x) + f(-x)}{2}$

	\begin{parts}
		\part Express $\hat{f}^{t}(S)$ in terms of $\hat{f}(S)$.
		\begin{solution}

		\end{solution}

		\part Verify that $f = f^{\text{even}} + f^{\text{odd}}$ and that $f$ is odd
		(respectively, even) if and only if $f = f^{\text{odd}}$.
	\end{parts}

	%\begin{solution}
	%{\bf Collaborators}:\\ 
	%Write solutions  afer uncommenting the lines above and below.
	%\end{solution}
	\question[10]
	\begin{parts}
		\part[5] Exercise 1.15 in the book Analysis of Boolean Functions by Ryan O'Donell.
		\part[5] Let $f:\{-1,1\}^n\to \{-1,1\}$ be a Boolean function. For $k\in [1,n]$,
		let $f^{=k}$ be the function given by $f^{=k}(x) = \sum_{S\subset [n], |S| = k}
			\widehat{f}(S) \chi_S(x)$. Prove the formula for $\langle f^{=k}, f^{=\ell}
			\rangle$ given in Exercise 1.18 of the book `` Analysis of Boolean Functions''
		by Ryan O'Donnel.
		\begin{solution}
			{\bf Collaborators}: None\\
			We need to show the following
			\begin{align*}
				\langle f^{=k}, f^{=l} \rangle =
				\begin{cases}
					W^{k}[f] \textsf{ if } k = l \\
					0 \textsf{ otherwise}
				\end{cases}
			\end{align*}

			\textbf{Definition} \textit{For $f: \{-1, 1\}^n \to \mathbb{R}$
				and $0 \leq k \leq n$ Fourier weight at degree $k$ is the following}

			\begin{align}
				W^k[f] = \displaystyle\sum_{\underset{S \subseteq [n]}{card(S) = k}} \hat{f}(S)^2
			\end{align}

			Using Definition (1) we can arrive at the inner product asked in the question.
			From Parseval's theorem we can say that $W^{k}[f] = ||f^{=k}||_2 ^{2}$. Where
			$f^{=k} = \sum_{card(S) = k} \hat{f}(S) \chi_{S}$.

			Now we calculate $\langle f^{=k}, f^{=l} \rangle$

			\begin{align}
				\langle f^{=k}, f^{=l} \rangle & = \left\langle \displaystyle\sum_{card(S) = k} \hat{f}(S) \chi_S, \displaystyle\sum_{card(T) = l} \hat{f}(T) \chi_T \right\rangle \\
				                               & = \displaystyle\sum_{card(S) = k} \displaystyle\sum_{card(T) = l} \hat{f}(S) \hat{f}(T) \langle \chi_S, \chi_T \rangle
			\end{align}

			From class we know that the quantity $\langle \chi_S, \chi_T \rangle = 1$ if $S
				= T$ hence $\langle \chi_S, \chi_T \rangle = 1$ \textit{iff} $card(S) = card(T)
				= k$.

			\vspace{1em}

			Continuing from (3)

			\begin{align}
				\langle f^{=k}, f^{=l} \rangle & = \displaystyle\sum_{card(S) = k} \hat{f}(S)^2
			\end{align}

			(4) is defined to be $W^k[f]$. Hence the following is proved.

			\begin{align*}
				\langle f^{=k}, f^{=l} \rangle =
				\begin{cases}
					W^{k}[f] \textsf{ if } k = l \\
					0 \textsf{ otherwise}
				\end{cases}
			\end{align*}

		\end{solution}
	\end{parts}

	\question[10]  We had stated in the class that  Knapsack problem admits a pseudo linear time algorithm. This question is on developing such an algorithm and using it for a simple approximation algorithm for Knapsack.
	\begin{parts}
		\part[5] Using the standard dynamic programing approach, develop a $O({\sf
					poly}(n)W)$ time algorithm for the Knapsack problem with $n$ items. Clearly
		mention each step and give a complexity analysis of the algorithm.

		\begin{solution}
			\textbf{Collaborators: }
		\end{solution}

		\part[5] Using the above algorithm, devlop an algorithm that given $\epsilon$,
		obtains an $1-\epsilon$ approximate solution(i.e., outputs a solution with
		profit at least (1-$\epsilon$) times the maximum) in time ${\sf
					poly}(n,1/epsilon)$.

		\begin{solution}
			\textbf{Collaborators: }
		\end{solution}
	\end{parts}

	\question[10]  Let $A$ be an algorithm for a Euclidean optimization problem ${\cal P}$. in the plane. Consider  an input instance $x$ and the simple perturbation model with a given parameter $\delta\in (0,1/2)$.  Let $x\in ([0,1]^2)^n$. A $\delta$ perturbation of $x$ is given by $(x_1+y_1,\ldots, x_n+y_n)$, where $y_1,\ldots, y_n$ are chosen uniformly and independent on $[-\delta, \delta]$.  The  algorithm  $A$ is said to be $(\epsilon,\delta)$ stable with respect to $x$, if $Pr[|A(x) - A(y)|\le  \epsilon] = 1-o(1)$, where $y$ is a $\delta$ perturbation of $x$.

	\begin{parts}
		\part[5] Consider the Christofide's algorithm for ETSP (Euclidean TSP) in the
		plane. Construct an infinite family of inputs $X= (X_n)_{n>0}$ (i.e., for every
		$n>0$, construct an instance $X_n$ with $O(n)$ many points) such that $CHR(X_n)
			= (1.5-o(n)) TSP(X))$.

		\begin{solution}
			\textbf{Collaborators: }
		\end{solution}

		\part[5] Are the instance you have constructed above $(\epsilon,\delta)$ stable for
		suitable values of $\delta$ and $\epsilon$? Give a detailed justification to
		your answer.

		\begin{solution}
			\textbf{Collaborators: }
		\end{solution}

	\end{parts}

	\question[10] This question uses the same notion of perturbation as in the above question.  Let ${\cal P}$ be a Euclidean optimization problem in the plane and $x\in (\mathbb{R})^2)^n$ be an input instance and $y$ be a $\delta$ perturbation of $x$.  We say that ${\cal P}$ is $(\epsilon,\delta)$ stable at $x$, if $Pr[{|\cal P}(x)-{\cal P}(y)| \le \epsilon] = 1-o(1)$.
	\begin{parts}
		\part[6] Suppose ${\cal P}$ be an $\NP$ hard Euclidean optimization problem such
		that $\forall x \in (\mathbb{R}^2)^n$, ${\cal P}$ is $(\epsilon, \delta)$
		stable at $x$, for some $\epsilon$ and $\delta$. Can we have a smoothed
		polynomial time algorithm that solves ${\cal P}$ exactly? Justify your answer.
		\part[4] Suppose $A$ is approximation algorithm for ${\cal P}$ with some
		approximation performance $\gamma$. Assume that $ X = (X_n)$ is an infinite
		family of worst case instances for $A$ (i.e., approximation ratio of $A$ on
		$X_n$ is $\gamma -o(1)$ ). Further, suppose that $X_n$ is $(\epsilon,\delta)$
		stable for ${\cal P}$ for every $n$. What will be the best possible smoothed
		approximation ratio of $A$ under $\delta$ perturbations? Prove your answers.
	\end{parts}
	\question[10] Consider the {\sf pathTSP} problem: Given a set of $n$ points  $x_1,\ldots, x_n \in [0,1]^2$, in the Euclidean plane, compute the minimum weight of a Hamiltonian path through the points.  Define a Euclidean functional corresponding to ${\sf pathTSP}$. Decide if the newly defined function satisfy any of the properties among  subadditivity, superadditivity, smoothness, growth bound. Give detailed arguments justifying your answer.
\end{questions}

\begin{center}
	{\bf Bonus}
\end{center}
\begin{questions}
	\question[5] Let $X$ be an instance of the $k$-means clusturing problem, with $|X| = n$. Let $B_1,\ldots, B_r$ be a sequence of clusturings obtained by the $k$-means algorithm. Show that there is some constant $c$ such that if $r>c$, then at least one of the clusters among the clusterings $B_1,\ldots, B_r$, has seen at least three different configurations. ({\bf Note:} in the class we had shown this for $r>2^k$, using the pigeonhole principle.)
\end{questions}
\end{document}