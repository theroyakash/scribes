\begin{frame}{Basic Definitions}
    \begin{itemize}
        \item In numerical problems it is easy to let an adversary come up with worst case example and then perturb the instance via some distribution (for example Gaussian Perturbation).
    \end{itemize}
\end{frame}

\begin{frame}{Basic Definitions}
    \begin{itemize}
        \item In numerical problems it is easy to let an adversary come up with worst case example and then perturb the instance via some distribution (for example Gaussian Perturbation).
        \item However for general problems this model can not be used.
    \end{itemize}
\end{frame}

\begin{frame}{Basic Definitions}
    \begin{itemize}
        \item In numerical problems it is easy to let an adversary come up with worst case example and then perturb the instance via some distribution (for example Gaussian Perturbation).
        \item However for general problems this model can not be used.
        \item From Beier Vöcking's model we'll let an adversary choose the whole probability distribution. Let's define the model more formally,
    \end{itemize}
\end{frame}

\begin{frame}{Beier Vöcking's One Step Model}
    \textit{Any input $X$ of length $n$ with $X = (x_1, \dots, x_n) \in F^n$ where $F$ is the domain, with parameter $\phi$ and an adversary who chooses density functions bounded by $\phi$ as $\{f_1, \dots, f_n\}$ such that $f_i:F\to [0, \phi]$, an algorithm $\mathcal{A}$'s smoothed performance measure given by the following}
    \begin{center}
        \begin{align*}
            \textit{smoothed performance }(\mathcal{A}) = \: \substack{\mathbb{E} \\ {X = (x_1, \dots, x_n), \: x_i \sim f_i: D_{n, x, \phi}}} \left[\mathcal{A}(X)\right]
        \end{align*}
    \end{center}
\end{frame}

\begin{frame}{Formal Definition of the Model}
    \begin{center}
        Our perturbation models are families of distribution $\mathcal{D} = (D_{n, \phi, x})$ where $n$ is the size of the input $x$, and $\phi$ is the upper bound on the maximum density of the probability distributions.
    \end{center}
\end{frame}

\begin{frame}
    \frametitle{Some Properties of Distribution}

    \begin{center}
        For every $n, x, \phi$ the support of the distribution should be
        of size $\{0,1\} ^{\leq \text{poly}(n)}$.
    \end{center}

\end{frame}


\begin{frame}
    \frametitle{$N_{n, x} \textsf{ and } S_{n, x}$}

    \begin{center}
        We define $N_{n, x} \text{ and } S_{n, x}$ here.
    \end{center}

\end{frame}


\begin{frame}
    \frametitle{$N_{n, x} \textsf{ and } S_{n, x}$}

    \begin{align*}
        S_{n, x} &= \{y \: \vert \: D_{n, x, \phi}(y) > 0 \text{ for some } \phi\} \\
        N_{n, x} &= \vert S_{n, x} \vert
    \end{align*}

\end{frame}


\begin{frame}{Some Properties of Distribution}
    \begin{itemize}
        \item We say $\mathcal{D}$ is parameterized by $n, \phi, x$.
    \end{itemize}
\end{frame}

\begin{frame}{Some Properties of Distribution}
    \begin{itemize}
        \item We say $\mathcal{D}$ is parameterized by $n, \phi, x$.
        \item For all $n, \phi, x$ and $y$ we demand $D_{n, \phi, x}(y) \leq \phi$,
        We choose $\phi$ $\in [\frac{1}{N_{n, x}}, 1]$ and $n$ is the size of input $x$.
    \end{itemize}
\end{frame}

\begin{frame}{Some Properties of Distribution}
    \begin{itemize}
        \item We say $\mathcal{D}$ is parameterized by $n, \phi, x$.
        \item For all $n, \phi, x$ and $y$ we demand $D_{n, \phi, x}(y) \leq \phi$,
        We choose $\phi$ $\in [\frac{1}{N_{n, x}}, 1]$ and $n$ is the size of input $x$.
        \item Choice of $\phi$ determines the strength of perturbation.
    \end{itemize}
\end{frame}

\begin{frame}{Some Properties of Distribution}
    \begin{itemize}
        \item We say $\mathcal{D}$ is parameterized by $n, \phi, x$.
        \item For all $n, \phi, x$ and $y$ we demand $D_{n, \phi, x}(y) \leq \phi$,
        We choose $\phi$ $\in [\frac{1}{N_{n, x}}, 1]$ and $n$ is the size of input $x$.
        \item Choice of $\phi$ determines the strength of perturbation.
        \item If we choose $\phi = 1$ this corrsoponds to worst case complexity and setting $\phi = \frac{1}{N_{x, \phi}}$ is average case complexity. 
    \end{itemize}
\end{frame}

\begin{frame}{Some Properties of Distribution}
    \begin{itemize}
        \item We say $\mathcal{D}$ is parameterized by $n, \phi, x$.
        \item For all $n, \phi, x$ and $y$ we demand $D_{n, \phi, x}(y) \leq \phi$,
        We choose $\phi$ $\in [\frac{1}{N_{n, x}}, 1]$ and $n$ is the size of input $x$.
        \item Choice of $\phi$ determines the strength of perturbation.
        \item If we choose $\phi = 1$ this corrsoponds to worst case complexity and setting $\phi = \frac{1}{N_{x, \phi}}$ is average case complexity. 
        \item The choice of $\phi$ must be discretized such that it can be represented within polynomial many bits.
    \end{itemize}
\end{frame}

\begin{frame}
    \frametitle{Smoothed Polynomial Running Time}

    \textbf{Definition 1} \textit{An algorithm $\mathcal{A}$ has smoothed polynomial
    running time with respect to the distribution family $\mathcal{D}$
    if there exists an $\epsilon > 0$ such that, for all $n, \phi, x$, we have}

    \begin{align*}
        \mathbb{E}_{y \sim D_{n, x, \phi}} \left(t_A(y; n, \phi) ^{\epsilon}\right) = O(nN_{n,x}\phi)
    \end{align*}
\end{frame}

\begin{frame}
    \frametitle{Analysing the Definition}
    
    \begin{center}
        Note that the up-above result do not speak about the expected running time, it takes
        into account for the $\epsilon$ moment of the expected running time.
    \end{center}
\end{frame}

\begin{frame}
    \frametitle{Analysing the Definition}
    
    \begin{center}
        This is because the expected running time is not robust.
    \end{center}
\end{frame}