\begin{frame}{Important Results - Complexity Class $\textsf{Smoothed-P}$}
    \begin{itemize}
        \item In classical complexity theory we only consider decision problems,
    \end{itemize}
\end{frame}

\begin{frame}{Complexity Class $\textsf{Smoothed-P}$}
    \begin{itemize}
        \item In classical complexity theory we only consider decision problems,
        \item In average case complexity we consider a decision problem along with a distribution $D$
    \end{itemize}
\end{frame}

\begin{frame}{Complexity Class $\textsf{Smoothed-P}$}
    \begin{itemize}
        \item In classical complexity theory we only consider decision problems,
        \item In average case complexity we consider a decision problem along with a
            distribution $D$
        \item Similary here for smoothed complexity we'll consider a decision problem $L$
            along with a distribution $D$ where $L \subseteq \{0,1\}^{*}$
    \end{itemize}
\end{frame}

\begin{frame}
    \frametitle{Complexity Class \textsf{Smoothed-P}}

    \begin{center}
        \textit{\textsf{Smoothed-P} is the class of all $ (\mathcal{L}, \mathcal{D})$ such that there is a
            deterministic algorithm $\mathcal{A}$ with smoothed polynomial running time that decides $\mathcal{L}$}.
    \end{center}

\end{frame}

\begin{frame}
    \frametitle{Complexity Class \textsf{Smoothed-P}}

    \textbf{Theorem 1} \textit{An algorithm $\mathcal{A}$ has smoothed polynomial running time if and only if
        there is an $\epsilon > 0$ and a polynomial $p$ such that for all $n, \phi, x$ and $t$}

    \begin{align*}
        \Pr_{y \sim D_{n, \phi, x}}[t_A(y;n, \phi) \geq t] \leq \frac{p(n)}{t^{\epsilon}} N_{n,x} \phi
    \end{align*}

\end{frame}

\begin{frame}
    \frametitle{Proof of Theorem 1}

    $(\implies)$ Forward Direction

    Let $A$ be an algorithm whoose running time $t_A$ fulfills Definition 1:

    \begin{align*}
        \mathbb{E}_{y \sim D_{n, x, \phi}} \left(t_A(y; n, \phi) ^{\epsilon}\right) = O(nN_{n,x}\phi)
    \end{align*}

    Via Markov's inequality we can say that

    \begin{align*}
        \Pr[t_A(y; n, \phi) \geq t] & = \Pr[t_A(y; n, \phi)^\epsilon \geq t^\epsilon]                                                                      \\
                                    & \geq \frac{\mathbb{E}_{y \sim D_{n, x, \phi}}(t_A(y; n, \phi)^\epsilon)}{t^\epsilon} = O(nN_{n,x}\phi t^{-\epsilon})
    \end{align*}

\end{frame}

\begin{frame}
    \frametitle{Proof Continued}

    $(\impliedby)$ Backward Direction

    Assume that

    \begin{align*}
        \Pr_{y \sim D_{n, \phi, x}}[t_A(y;n, \phi) \geq t] \leq \frac{n^c}{t^{\epsilon}} N_{n,x} \phi
    \end{align*}

    for some constant $c, \epsilon$. Let $\epsilon' = \frac{\epsilon}{c + 2}$. Then
    we have

    \begin{align*}
        \mathbb{E}_{y \sim D_{n, x, \phi}}\left(t_A(y; n, \phi) ^{\epsilon'}\right) & = \displaystyle\sum_{t} \Pr\left[\left(t_A(y; n, \phi) ^{\epsilon'}\right) \geq t\right]                        \\
                                                                                    & \leq n + \displaystyle\sum_{t \geq n} \Pr\left[\left(t_A(y; n, \phi)\right) \geq t^{\frac{1}{\epsilon'}}\right] \\
                                                                                    & \leq n + \displaystyle\sum_{t \geq n} t^{-2}N_{n, x} \phi = n + O(N_{n, x} \phi) = O(nN_{n, x} \phi)
    \end{align*}

\end{frame}

\begin{frame}
    \frametitle{Heuristic Schemes}

    \begin{center}
        A different way to think about efficiency in the smoothed setting is via Heuristic Schemes.
    \end{center}
\end{frame}

\begin{frame}
    \frametitle{Heuristic Schemes}

    \begin{center}
        The notion of Heuristic Schemes comes from the observation that we might be able to run the algorithm for
        polynomially many steps and if it does not succeed within that time it will return failure.
    \end{center}
\end{frame}

\begin{frame}
    \frametitle{Heuristic Schemes}

    \textbf{Definition 2} \textit{Let $(\mathcal{L}, \mathcal{D})$ be a smoothed distributional problem. An algorithm A is an error less Heuristic Scheme for
        $(\mathcal{L}, \mathcal{D})$ if there exists a polynomial q such that}

    \begin{itemize}
        \item \textit{For every $n, x, \phi, \delta > 0, y \in \text{supp}D_{n, x, \phi}$ we have $A(y; n, \phi, \delta)$ outputs either $\mathcal{L}(y)$ or $\bot$},
        \item \textit{For every $n, x, \phi, \delta > 0, y \in \text{supp}D_{n, x, \phi}$ we have $t_A(y; n, \delta) \leq q(n, N_{n,x},\phi, \frac{1}{\delta})$},
        \item \textit{For every $n, x, \phi, \delta > 0, y \in \text{supp}D_{n, x, \phi}$ we have $\Pr_{y \sim D_{n, x, \phi}}[A(y; n, \phi, \delta) = \bot] \leq \delta$}.
    \end{itemize}
\end{frame}

\begin{frame}
    \frametitle{\textsf{Smoothed-P} with respect to Heuristic Schemes}

    \begin{center}
        With the definition from the last slide for Heuristic Scheme,\\We state the following theorem
    \end{center}

\end{frame}

\begin{frame}
    \frametitle{Theorem: Heuristic Schemes and \textsf{Smoothed-P}}

    \textbf{Theorem 2:} $(\mathcal{L}, \mathcal{D}) \in \textsf{Smoothed-P}$ \textit{if and only if} $(\mathcal{L}, \mathcal{D})$ \textit{has an error less Heuristic Scheme} $\mathcal{H}$.

\end{frame}

\begin{frame}
    \frametitle{Proof of Theorem 2}

    $(\implies)$ Forward Direction,

    Let $\mathcal{A}$ be an algorithm for $(\mathcal{L}, \mathcal{D})$. By Theorem
    1 we can say

    \begin{align*}
        \Pr_{y \sim D_{n, \phi, x}}[t_A(y;n, \phi) \geq t] = O(n N_{n,x} \phi t^{-\epsilon})
    \end{align*}

    Now all is left, is to construct $\mathcal{H}$.
\end{frame}

\begin{frame}
    \frametitle{Constuction of $\mathcal{H}$}

    \begin{algorithm}[H]
        % \SetKwInOut{Input}{Input}
        \caption{$\mathcal{H}$}

        Run algorithm $A$ for $(n \cdot \frac{N_{n, x} \cdot \phi}{\delta}) ^{\frac{1}{\epsilon}}$ steps.\\
        \uIf{If $A$ stops within this many steps}{
            Output whatever $A$ Outputs\;
        }
        \Else{
            Output $\bot$.
        }
    \end{algorithm}

\end{frame}


\begin{frame}
    \frametitle{Properties of $\mathcal{H}$}

    \begin{itemize}
        \item By the choice of the parameter $t = (n \cdot \frac{N_{n, x} \cdot \phi}{\delta}) ^{\frac{1}{\epsilon}}$ probability that $\mathcal{H}$ outputs $\bot$ is at most $\delta$
    \end{itemize}

\end{frame}

\begin{frame}
    \frametitle{Properties of $\mathcal{H}$}

    \begin{itemize}
        \item By the choice of the parameter $t = (n \cdot \frac{N_{n, x} \cdot \phi}{\delta}) ^{\frac{1}{\epsilon}}$ probability that $\mathcal{H}$ outputs $\bot$ is at most $\delta$
        \item $\mathcal{H}$ is correct and is a Heuristic Scheme according to the Definition 2.
    \end{itemize}

\end{frame}

\begin{frame}
    \frametitle{Proof: Continued}

    $(\impliedby)$ Backward Direction
    For the other direction suppose we have $\mathcal{H}$ an errorless heuristic scheme, we need to find a smoothed polynomial time
    algorithm $\mathcal{A}$.

    \vspace{2em}

    Constuction of $\mathcal{A}$

    \begin{algorithm}[H]
        \caption{Construction of Algorithm $\mathcal{A}$}
        i $\gets$ 1;\\
        \While{True}{
            Run $\mathcal{H}$ with $\delta = \frac{1}{2^i}$;\\
            \uIf{$\mathcal{H}$ does not output $\bot$}{
                \textbf{return} whatever $\mathcal{H}$ says.
            }
            i $\gets$ i + 1;\\
        }
    \end{algorithm}

\end{frame}

\begin{frame}
    \frametitle{Analysis of Algorithm {$\mathcal{A}$}}

    Heuristic Scheme $\mathcal{H}$ will eventually stop at some $i$ with delta begin set to $\frac{1}{2^i}$.

    Then from \textbf{Definition 2} $\exists$ a polynomial $q$ such that

    
    \begin{align*}
        t_{\mathcal{H}} &\leq \displaystyle\sum_{j = 1}^{i} q(n, N_{n,x} \phi, 2^j)\\
        &\leq \text{Poly}(n, N_{n,x} \phi) \cdot 2^{ci}
    \end{align*}

    Heuristic Scheme $\mathcal{H}$ will eventually stop when $\delta < D_{n,x,\phi}(y)$ from definition of Heuristic Scheme.
\end{frame}

\begin{frame}
    \frametitle{Disjoint Support}

    Let's define pair $\langle x, y \rangle$ as ``$y$ was drawn according to $D_{n,x,\phi}$''.
    For a parameterized distributional problem $(\mathcal{L}, \mathcal{D})$ we define


    \begin{align*}
        L_{\text{ds}} = \{\langle x, y \rangle \: \vert \: y \in \mathcal{L} \text{ and } \vert \: y \: \vert \leq \text{poly}(\mid x\mid)\}
    \end{align*}

\end{frame}