\documentclass[12pt, a4paper]{article}
\usepackage{scribe}
\usepackage{amsmath}
\usepackage{amssymb}
\usepackage{graphicx}

\setlength\parindent{0pt}

\usepackage[linesnumbered,ruled]{algorithm2e}
\graphicspath{ {./images/} }

\usepackage{tcolorbox}
\tcbuselibrary{theorems}

\newtcbtheorem[number within=section]{theo}{Theorem}%
{colback=green!5,colframe=green!35!black,fonttitle=\bfseries}{th}

\newtcbtheorem[number within=section]{lemma}{Lemma}%
{colback=yellow!30,colframe=red!75!black,fonttitle=\bfseries}{th}

\newtcbtheorem[number within=section]{hypothesis}{Hypothesis}%
{colback=red!5,colframe=red!35!black,fonttitle=\bfseries}{th}

\newtcbtheorem[number within=section]{define}{Definition}%
{colback=blue!5,colframe=blue!35!black,fonttitle=\bfseries}{th}

% RANDOM COMMANDS
\newcommand{\tsp}{{\sf TSP}}
\newcommand{\mst}{{\sf MST}}
\newcommand{\mm}{{\sf MM}}
\newcommand{\mwt}{{\sf MWT}}
\newcommand{\E}{\mathbb{E}}
\newcommand{\ch}{{\sf CH}}

\newcommand{\distras}[1]{%
    \savebox{\mybox}{\hbox{\kern3pt$\scriptstyle#1$\kern3pt}}%
    \savebox{\mysim}{\hbox{$\sim$}}%
    \mathbin{\overset{#1}{\kern\z@\resizebox{\wd\mybox}{\ht\mysim}{$\sim$}}}%
}


% DETAILS
\Scribe{Roy, Akash}
\Lecturer{Raghavendra Rao, B. V.}
\LectureNumber{26}
\LectureDate{March 13, 2023}
\LectureTitle{Improved Bound on 2-{\sf OPT} Algorithm}


\begin{document}
\MakeScribeTop

\textbf{Topics}: Improved bound on the 2-{\sf OPT} algorithm for $\tsp$ tour.

\section{Improved Analysis}

Last class we looked at the upper bound on the running time of 2\textsf{-OPT} algorithm 
is $m^2n^2\phi \lg n$. In this lecture we'll try to improve this bound. In order to do this 
we need the concept of linked $2$ exchanges. The idea is we start with some tour, be it 
Christofides or Double tree, then we start 2-{\sf OPT} exchanges. Now let's suppose
$(S, \dots, S_t)$ be any sequence of exchanges. Suppose $S_i$ and $S_j$ be any two exchanges in the
sequence where $S_i$ happend before $S_j$. We say $(S_i, S_j)$ are linked
if there exists some edge $e_i$ added in $S_i$ and got removed in $S_j$.

\begin{lemma}{Lower bound on the number of linked disjoint pairs}{lemma1}
    \textit{In any sequence of $t$ many exchanges by 2-OPT algorithm there are at least}
    $\frac{2t - n}{7}$ \textit{ disjoint pairs that are linked.}
\end{lemma}

\textbf{Proof.} Suppose $(S, \dots, S_t)$ be any sequence of exchanges.
Iterativly construct a set $\mathcal{L}$ of linked 2 exchanges. Now construct
$\mathcal{L}'$ from $\mathcal{L}$ by removing intersecting paris. Now
$\mathcal{L}'$ has only disjoint pairs. Disjoint pair $(S_i, S_j), (S_i', S_j')$ 
is such that $S_i \neq S_j', S_i'$ and $S_j \neq S_j', S_i'$.

\textbf{Observation.} Every pair $(S_i, S_j)$ is disjoint from $\mathcal{L}$
except $6$ pairs. So

\begin{align*}
    \vert \mathcal{L}' \vert &\geq \frac{\vert \mathcal{L} \vert}{7}\\
    &\geq \frac{2t - n}{7}\\
    &\geq \frac{t}{4}
\end{align*}

\begin{lemma}{Estimation Improvement in 2-OPT}{lemma2}
    \textit{In $\phi$ purturbed graphs of n vertices and m edges, the probability that there exists
    a pair of linked $2$ exchanges with improvements by at most $\epsilon$ is} $O(m^3 \epsilon^2 \phi^2)$
\end{lemma}

\textbf{Proof.} Let $(S, S')$ be a linked pair. $S$ has the following edges $\{e_1, e_2, e_3, e_4\}$
and $S'$ has the following edges $\{e_3, e_5, e_6, e_7\}$. Suppose in the first exchange edeges $e_1, e_2$
are being exchanged with $e_3, e_4$, in the second exchange edeges $e_4, e_5$, are exhanged with edges
$e_6, e_7$. We need to bound the probability of $\Delta(e_1, e_2, e_3, e_4) \leq \epsilon$ and $\Delta(e_3, e_5, e_6, e_7) \leq \epsilon$. We'll use the principle of deffered decision to bound the probability. 
Probability of $\Delta\{e_1, e_2, e_3, e_4\} \in (0, \epsilon)$ is equivalent to the probability of $\textsf{dist}(e_1) \in (\kappa, \kappa + \epsilon)$ and $\Delta\{e_4, e_5, e_6, e_7\} \in (0, \epsilon)$
is same as the probability of $\textsf{dist}(e_6) \in (\kappa_2 - \epsilon, \kappa_2)$. Here $e_1$ is the edge going out and $e_6$ is edge coming in and
$\kappa = \textsf{dist}(e_3) + \textsf{dist}(e_4) - \textsf{dist}(e_2)$ and $\kappa_2 = \textsf{dist}(e_4) + \textsf{dist}(e_5) - \textsf{dist}(e_7)$.

$\textsf{dist}(e_1)$ and $\textsf{dist}(e_6)$ are random variables with densities bound from above
by $\phi\epsilon$. This makes $\Pr[\textsf{dist}(e_1) \in (0, \epsilon) \wedge \Pr[\textsf{dist}(e_6)] \in (0, \epsilon)] \leq \phi^2\epsilon^2$.
There are at most $O(m^3)$ number of linked 2-exchanges, so via union bound we can conclde the proof of lemma
1.2 that $\Pr[\Delta^* \leq \epsilon] \leq m^3\epsilon^2\phi^2$.

Based on the Estimation Improvement in 2-OPT lemma and Lower bound on the number of linked disjoint pairs
lemma we can state the following theorem.

\begin{theo}{Estimation Improvement in 2-OPT}{theorem1.1}
    \textit{In $\phi$ purturbed graphs of n vertices and m edges, excepted length of the longest path
    in the 2-\textsf{OPT} configuration graph is} $O(m^{\frac{3}{2}} n \phi)$
\end{theo}

\textbf{Proof:} Suppose $\tau$ be the longest path in the configuration graph of the
2-\textsf{OPT} algorithm. Let $\Delta^*$ be the smallest improvement during any pair of linked 2 exchanges.

If $T \geq t$ then there must be a sequence of $t$ 2-exchanges in the configuration graph of the 2-\textsf{OPT} algorithm.
From lemma 1.1 we can say that there is at least $z = \frac{2t - n}{7}$ many pairs. From lemma 1.1

\begin{align*}
    \Pr[T \geq t] &\leq \Pr\left[\Delta^* \leq \frac{4n}{t}\right] \leq O(m^3 t^{-2} \phi^2 n^2)\\
    \mathbb{E}[T] &= \displaystyle\sum_{t = 1} ^{n!} \Pr[T \geq t]\\
    &\leq \displaystyle\sum_{t = 1} ^{n!} \min(1, c_1 m^3 t^{-2} \phi^2 n^2)\\
    &\leq \displaystyle\sum_{t = 1} ^{n \phi m^{3/2}} 1 + \displaystyle\sum_{t = n \phi m^{3/2} + 1} ^{n!} c_1 m^3 t^{-2} \phi^2 n^2\\
    &\leq m^{3/2} n \phi + \int_{n \phi m^{3/2} + 1}^{\infty} c_1 m^3 t^{-2} \phi^2 n^2 dt\\
    &= m^{3/2} n \phi + \left[ - c_1m^3n^2 \phi \right]_{m^{3/2}n^2 \phi} ^{\infty}\\
    &= O(m^{\frac{3}{2}}n \phi)
\end{align*}

\end{document}