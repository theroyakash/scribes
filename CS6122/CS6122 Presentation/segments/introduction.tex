\begin{frame}{Introduction}
    \begin{itemize}
        \item Goal of the complexity theory is to understand computational difficulty of engineering problems.
    \end{itemize}
\end{frame}

\begin{frame}{Introduction}
    \begin{itemize}
        \item Goal of the complexity theory is to understand computational difficulty of engineering problems.
        \item So we've developed theory to classify problems according to their worst case behaviour. These classes are \textsc{P}, \textsc{NP} etc.
    \end{itemize}
\end{frame}

\begin{frame}{Introduction}
    \begin{itemize}
        \item Goal of the complexity theory is to understand computational difficulty of engineering problems.
        \item So we've developed theory to classify problems according to their worst case behaviour. These classes are \textsc{P}, \textsc{NP} etc.
        \item \textsc{P} class contains all the computational problems that in the worst case completes in polynomial time with respect to the size of the input.
    \end{itemize}
\end{frame}

\begin{frame}{Introduction}
    \begin{center}
        In our \textsc{CS6122} Course we've already seen that real world instances for few NP-Complete problems performs \textbf{good} in terms of running time.
    \end{center}
\end{frame}

\begin{frame}{Introduction}
    \begin{center}
        Thus we must develop theory that'll classify problems of their computational difficulty \textbf{with respect to real world performance} as well. Thus we develop smooth complexity theory.
    \end{center}
\end{frame}

\begin{frame}{Topics we'll look into}
    In this presentation we'll look into the following
    \begin{itemize}
        \item Basic Definitions and assumptions, $\textsf{Smoothed-P}$ Class.
            \begin{itemize}
                \item 2-step vs 1-step model, why we are using one step models,
                \item Support of the distribution, notion of $N_{x, n}$.
                \item Concept of Family of Distribution
                \item Definition of smoothed polynomial running time \textbf{Definition 2.1},
                \item Definition of $\textsf{Smoothed-P}$
                \item \textbf{Theorem 2.3} \textit{An algorithm A has smoothed polynomial running time if and only if there is an} $\epsilon > 0$ \textit{and a polynomial} $p$ \textit{such that for all n, x, $\phi$ and t} $$\Pr_{y \sim D_{n, \phi, x}}[t_A(y; n, \phi) \geq t] \leq \frac{p(n)}{t^\epsilon} N_{n, x} \phi$$
            \end{itemize}
    \end{itemize}

\end{frame}

\begin{frame}{Topics - Continued}
    \begin{itemize}
        \item Heurisitic Schemes, error less heuristic schemes in $\textsf{Smoothed-P}$.
        \item Notion of Reduciblity, define $L_{ds}$, notion of completeness.
            \begin{itemize}
                \item Distributional problems
                \item Polynomial time smoothed reductions $\leq_{smoothed}$
                \item Transitivity of $\leq_{smoothed}$ [via theorem 3.4]
                \item Theorem 3.5 $(L, D) \in \textsf{Smoothed-P}$ if and only if $(L_{ds}, D) \in \textsf{Smoothed-P}^{\textsf{obl}} _{ds}$
            \end{itemize}
    \end{itemize}
\end{frame}


\begin{frame}{Topics - Continued}
    \begin{itemize}
        \item Parameterized Distributional NP $\textsf{Dist-NP}_{\text{para}}$
            \begin{itemize}
                \item Introduction
                \item One $\textsf{Dist-NP}_{\text{para}}$ complete problem to show $\textsf{Dist-NP}_{\text{para}}$ has complete problems
                \item \textbf{Weird looking theorem} \textit{$(\textsf{Tiling}, U^{\textsf{Tiling}})$ is $\textsf{Dist-NP}_{\text{para}}$-complete for some $U^{\textsf{Tiling}} \in \textsf{PComp}_{para}$ under polynomial time smoothed reductions}.
            \end{itemize}

        \item Basic Relation to Worst case complexity
        \item Notion of unsatisfiability and $\textsf{Smoothed-RP}$
        \item Theorem $k \textsf{UNSAT}_{\beta} \in \textsf{Smoothed-RP}$ for $\beta = \Omega(\sqrt{n \log \log n})$
        \item Concluding remarks
    \end{itemize}
\end{frame}