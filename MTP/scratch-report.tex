\documentclass{article}

\usepackage{arxiv}

\usepackage[utf8]{inputenc} % allow utf-8 input
\usepackage[T1]{fontenc}    % use 8-bit T1 fonts
\usepackage{hyperref}       % hyperlinks
\usepackage{url}            % simple URL typesetting
\usepackage{booktabs}       % professional-quality tables
\usepackage{amsfonts}       % blackboard math symbols

\usepackage{graphicx}
\graphicspath{ {./images/} }

\usepackage{caption}
\usepackage{subcaption}
\usepackage{float}
\usepackage[table,xcdraw]{xcolor}
\usepackage[linesnumbered,ruled]{algorithm2e}

\usepackage{tikz}

% \usepackage{tcolorbox}
% \tcbuselibrary{theorems}

% \newtcbtheorem[number within=section]{theo}{Theorem}%
% {colback=green!5,colframe=green!35!black,fonttitle=\bfseries}{th}

% \newtcbtheorem[number within=section]{lemma}{Lemma}%
% {colback=yellow!30,colframe=red!75!black,fonttitle=\bfseries}{th}

% \newtcbtheorem[number within=section]{hypothesis}{Hypothesis}%
% {colback=red!5,colframe=red!35!black,fonttitle=\bfseries}{th}

% \newtcbtheorem[number within=section]{define}{Definition}%
% {colback=blue!5,colframe=blue!35!black,fonttitle=\bfseries}{th}

\usepackage{amsmath}
\usepackage{amsthm}
\newtheorem{theorem}{Theorem}[section]

\newtheorem{define}{Definition}[section]

\newtheorem{hypothesis}{Hypothesis}
\newtheorem{lemma}[theorem]{Lemma}

\theoremstyle{remark}
\newtheorem*{remark}{Remark}

\title{Design and Analysis of Algorithms for Packing Coloring}

\author{
    theroyakash\thanks{\textsf{Department of Computer Science, submitted for course} \textsf{MTP}, \href{https://theroyakash.com}{\textsf{theroyakash.com}}}\\
    Indian Institute of Technology Madras \\
    \texttt{hey@theroyakash.com, cs22m007@smail.iitm.ac.in}
}

\begin{document}
\maketitle

\begin{abstract}
    In this report I discuss mathematical properties of a
    greedy heuristic I developed for the graph packing coloring problem.
    I discuss the design of an approximate coloring scheme for the packing coloring
    and explore its properties, compare approximation quality 
    and discuss novelity in implementation, and scalability.
\end{abstract}

\section{Introduction}
\subsection{Priliminary}
First we understand what is graph packing-coloring?

\begin{define}{\textsf{$S$-Packing Coloring and Packing Coloring}:}
    \textit{Suppose $S = (a_i)_{i \in \left[1 \to \infty\right)}$ is a increasing sequence of integers,
    then $S$ packing coloring of the graph is partition on the vertex set
    $V(G)$ into sets $V_1, V_2, V_3 \dots$ such that for every pair $(x, y) \in V_k$
    is at a distance more than $a_k$. If $a_i = i$ for every 
    $i \in \left[1 \to \infty\right)$, then we call the problem packing coloring.}.
\end{define}

\vspace{1mm}

\begin{define}{\textsf{$S$-Packing Chromatic Number}:}
    \textit{If there exists an integer $k$ such that $V(G) = V_1, V_2, V_3 \dots V_k$,
    each $V_i$ is a vertex-partition, then this partition is called
    $S$-packing, $k$ coloring, and minimum of such $k$ is the $S$-Packing Chromatic Number.}
\end{define}

Below I state an greedy heuristic that computes a valid coloring. 
The output (number of colors used)
however is \textit{an approximation} of the packing chromatic number.

\section{Greedy Heuristic For Packing Coloring}
Before we look at the heuristic on general (or random) graphs,
we'll look into the behaviour of the algorithm for special graphs.

First we look into complete ternary trees.

\subsection{Greedy Heuristic For Packing Coloring in complete 3-ary trees}
Following is the most naive implementation for the three ary packing coloring.
This is not the optimal, there are several optimizations we can do,
which we'll discuss later.

\begin{algorithm}[H]\label{alg:1}
    \SetKwInOut{Input}{Input}

    \Input{Tree $T$}
    \BlankLine
    Compute Level order traversal of Tree $T$\;
    Color Every Odd layer nodes with \textbf{COLOR(1)}\;
    \texttt{level} $\gets n - 2$\;
    \While{\texttt{level} $\geq 0$}{
        \texttt{maximum\_permissible\_color} = $n$\;
        \texttt{current\_color} = $2$\;
        \ForEach{\textsf{Node} in this \textsf{level}}{
            \While{\texttt{current\_color} $<$ \texttt{maximum\_permissible\_color}}{
                Travel to every node within distance \texttt{( int ) current\_color} and check if there is any node
                colored with color \texttt{current\_color}\;

                \uIf{None of the node is colored with color \texttt{current\_color}} {
                    Color this node with color \texttt{current\_color}\;
                    \textbf{break from the loop, go to next node in level}\;
                }\Else{
                    \texttt{current\_color} $\gets$ \texttt{current\_color} $+1$\;
                }
            }
        }

        \textsf{level} $\gets$ \textsf{level} $-1$\;
    }

    \textbf{Output:} Output the coloring of the graph.

    \caption{\textsc{Basic Greedy Algorithm}}
\end{algorithm}

\section{Optimizations}
This algorithm is not optimal, we can do several optimizations, below
we discuss all of them one after the other.

\subsection{Optimization 1}


\end{document}